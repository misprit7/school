\documentclass[letterpaper, reqno,11pt]{article}
\usepackage[margin=1.0in]{geometry}
\usepackage{color,latexsym,amsmath,amssymb,graphicx, float}
\usepackage{hyperref}

\hypersetup{
colorlinks=true,
linkcolor=magenta,
filecolor=magenta,
urlcolor=cyan,
}

\graphicspath{ {images/} }

\begin{document}
\pagenumbering{Arabic}
\title{PHYS 400 Homework 1}
\date{12/01/23}
\author{Xander Naumenko}
\maketitle

{\noindent\bf Question a.} Let $z=x+iy$, $z_1=x_1+iy_1$ and $z_2=x_2+iy_2$ for all the below. Then we have: 
\[
    \frac{1}{2}\left( z+z* \right)=\frac{1}{2}\left( x+iy+x-iy \right)=x=\text{Re } z
.\]

\[
    \frac{1}{2i}\left( z-z* \right)=\frac{1}{2}\left( x+iy-x+iy \right)=y=\text{Im } z
.\]

\[
    \left( \text{Re }z_1 \right) \left( \text{Re } z_2 \right) -\left( \text{Im }z_1 \right) \left( \text{Im }z_2 \right) =x_1x_2-y_1y_2=\text{Re }\left( x_1x_2-y_1y_2+x_1y_2i+x_2y_1i \right) =\text{Re }\left( z_1z_2 \right) 
.\]

{\noindent\bf Question b.} Again using $z=x+iy$:

 \[
\text{Im }\left| z^2 \right| =\text{Im }\left( x^2+ixy-ixy+y^2 \right)=0
.\]
\[
\text{Im }z^2=\text{Im }\left( x^2+2ixy-y^2 \right)=2xy
.\]

{\noindent\bf Question c.} Expanding using the Taylor series for the exponential and recalling those for the trigonometric functions: 

\[
e^{ix}= \sum_{n=0}^{\infty} \left( -1 \right)^{n} \frac{x^{n}}{n!}=\sum_{n\text{ odd}}^{\infty} \left( -1 \right)^{n} \frac{x^{n}}{n!}+\sum_{n\text{ even}}^{\infty} \left( -1 \right)^{n} \frac{x^{n}}{n!}=\cos x + i\sin x
.\]

{\noindent\bf Question d.} 
\[
    \text{Re }z=\text{Re }\left( A\cos \theta + iA\sin \theta \right) =A\cos \theta
.\]
\[
\text{Im }z=\text{Im }\left( A\cos \theta + iA\sin \theta \right) = A\sin \theta
.\]
\[
z*=A\cos\theta-A\sin\theta=Ae^{-i\theta}
.\]
\[
    |z|=\sqrt{ zz*}=\sqrt{A^2e^{-i\theta}e^{i\theta}} =A
.\]

{\noindent\bf Question e.} Expanding: 
\[
e^{i\alpha}e^{i\beta}=\cos\alpha\cos\beta+i\cos\beta\sin\alpha+i\cos\alpha\sin\beta-\sin\alpha\sin\beta=e^{i(\alpha+\beta)}=\cos\left( \alpha+\beta \right) +i\sin\left( \alpha+\beta \right) 
.\]

Taking the real and imaginary parts of the two sides gives us the standard trig identities: 

\[
\cos(\alpha+\beta)=\cos\alpha\cos\beta-\sin\alpha\sin\beta
.\]
\[
\sin(\alpha+\beta)=\sin\alpha\cos\beta+\cos\alpha\sin\beta
.\]

\end{document}
