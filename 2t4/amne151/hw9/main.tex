\documentclass[letterpaper, reqno,11pt]{article}
\usepackage[margin=1.0in]{geometry}
\usepackage{color,latexsym,amsmath,amssymb,graphicx,float,listings,tikz}
\usepackage{hyperref}

\hypersetup{
colorlinks=true,
linkcolor=magenta,
filecolor=magenta,
urlcolor=cyan,
}

\graphicspath{ {images/} }
\usepackage{setspace}
\doublespacing

\begin{document}
\pagenumbering{arabic}
\title{ANNE Close Reading 9}
\date{10/02/23}
\author{Xander Naumenko}
\maketitle

{\bf Question:} How does this text understand death? What does it tell us about what happens when someone dies?

\medskip

Euripides' play {\em Heracles} is a surprisingly candid view of the ancient Greek view on suicide that relates surprisingly well to our modern day view on it. The play is presented primarily as a dialogue between Heracles, Amphitryon and Theseus in the direct aftermath of Heracles' Hera-induced killing spree, with a Heracles and Theseus giving sizeable monologues at the end. The discussion between Heracles and Theseus in the second half is specifically relevant to the topic of suicide.

\medskip

It seems that Heracles is initially convinced that suicide is the only honorable route for one in his shoes. When Theseus asks what he will do, Heracles responds ``I will die and return to that world below from which I have just come.'' (line 1240) Perhaps even more interesting though is when Theseus responds soon after with ``Are these indeed the words of Heracles, the much-enduring?'' (Line 1250) The interpretation here isn't completely clear of course, that section reads surprisingly similar to how someone today might support a friend who is at risk of self harm. Heracles is understandably distraught from the realization he has (through no fault of his own) killed his family, and Theseus encourages him to calm himself and realize his self worth. Especially in contrast with many of the other texts we've read this term where the characters are all boastful and rather shallow, this section was surprisingly relatable and candid.

\medskip

In {\em Heracles} the cause of Heracles' grief is somewhat surprising from a modern perspective. After learning he killed his wife and children, he laments his loss repeatedly. However, he seems to regret more the change to his social standing that will inevitably follow his misdeeds more than the intrinsic sadness of the loss of life. This is best shown in his brief monologue before Theseus arrives, where he describes his regret for ``the infamy which now awaits [him]?'' (Line 1145) He focuses far more on how he is afraid society will see him in wake of the debacle if he doesn't choose the noble choice of death, and is driven to thoughts of suicide by a twisted sense of honor than one of direct sadness. This is in contrast to what I at least understand the current social understanding behind taking ones own life. While there are certainly cultures in which honor-based suicide is still an issue (Japan comes to mind), the modern, potentially inaccurate stereotype of suicidal individuals is that of someone chronically depressed for intrinsic reasons. 

\medskip

Altogether {\em Heracles} was a thought provoking play providing good insight into the ancient Greek view on an important issue. It's difficult to read between the lines in many of the other references to death in some of the other close reading texts we've seen to get a better understanding of it, so a text that is so clear on what a hero and a mortal might perceive the issue is especially interesting.

\end{document}
