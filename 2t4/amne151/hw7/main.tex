\documentclass[letterpaper, reqno,11pt]{article}
\usepackage[margin=1.0in]{geometry}
\usepackage{color,latexsym,amsmath,amssymb,graphicx,float,listings,tikz}
\usepackage{hyperref}

\hypersetup{
colorlinks=true,
linkcolor=magenta,
filecolor=magenta,
urlcolor=cyan,
}

\graphicspath{ {images/} }
\usepackage{setspace}
\doublespacing

\begin{document}
\pagenumbering{arabic}
\title{ANNE Close Reading 6}
\date{10/02/23}
\author{Xander Naumenko}
\maketitle

{\bf Question:} How does this text understand "gender" or "sex" differences? What does it tell us about these differences? 

{\em Note: ideas for this close reading were inspired by discussion D12, March 10.}

Our primary and secondary readings from this week point out a relative imbalance in the gender of traditional trickster gods. The primary reading this week, Homeric Hymn 4 ``To Hermes'' obviously focuses on the male trickster Hermes, but the other tricksters from the other primary texts we've seen thus far, such as Dionysus and Prometheus, have also been male. This is true not just in Greek mythology; Loki and Set from Norse and Egyptian mythology respectively are both male. 

What's interesting is that it's difficult to pinpoint exactly why this is from any primary text. As a child in Homeric Hymn 4, Hermes' gender doesn't play a large role in the story. Ironically, if anything the way that many of the Greek primary texts thus far have portrayed women would have predisposed them to a role as a trickster Godess. For example in most of Hesiod's texts from previous weeks, especially with reference to Pandora, he makes it clear that women are sly and trick men in various ways. Despite this, the trickster god role seems to be tied inexplicably to gods rather than goddesses. Whether this is merely coincidence or portrays a more fundamental insight into the functioning of ancient mythologies would require more research, but it's an interesting point to consider given the apparent gender imbalance among trickster gods. 


\end{document}
