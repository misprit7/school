\documentclass[letterpaper, reqno,11pt]{article}
\usepackage[margin=1.0in]{geometry}
\usepackage{color,latexsym,amsmath,amssymb,graphicx,float,listings,tikz}
\usepackage{hyperref}

\hypersetup{
colorlinks=true,
linkcolor=magenta,
filecolor=magenta,
urlcolor=cyan,
}

\graphicspath{ {images/} }
\usepackage{setspace}
\doublespacing

\begin{document}
\pagenumbering{arabic}
\title{ANNE Close Reading 4}
\date{10/02/23}
\author{Xander Naumenko}
\maketitle

{\bf Question:} How does this text understand political power? What does it tell us about who is considered the powerless or disenfranchised?   

Homeric Hymn 2 To Demeter is an insightful look into the power balance between the various Olympians plays out, where even traditionally less powerful deities hold power over the entire fate of mankind and by extension the other gods. 

\medskip 

When Demeter goes into mourning for the loss of her daughter the agriculture of the humans ceases, and the continued existence of the gods' sacrifices is put into question. When the other gods realize this, they immediately come to Demeter to appeal her to let grain sprout once more. When Iris comes to Demeter she says ``father Zeus, whose wisdom is everlasting, calls you to come join the tribes of the eternal gods.'' Although Zeus doesn't visit her himself, she clearly holds some form of power over him as he repeatedly sends messengers pleading her to help the humans. Eventually he must at least partially give into her demands and ask Hades to return Persephone. 

\medskip

This is interesting from a power dynamics perspective because even though it is made clear that no other god rivals Zeus in terms of raw might and power, he is not immune to the whims of the other gods and goddesses. Even the god of something such as agriculture, which on the face of it is not a threatening domain to have mastery over, can impact the world at large through indirect effects such as the mortal worshiper's devotion. 

\medskip

Despite the threat of Demeter's grief, it's interesting that Zeus at no point makes an actual direct appearance in the text giving direct orders. He only sends messengers at various points, with Iris, Hermes and Rhea all at various points carrying out Zeus's will. The implications of this are not obvious, but one interpretation of this is that Zeus see's this whole debacle as beneath him. He clearly has an interest in preserving humanity, but he does not deem it worthy of his direct intervention. It's also possible that he is trying not to appear desperate in front of the other gods, as coming down from Olympus to effectively beg Demeter to cooperate would potentially be damaging to his fearsome reputation. Whatever the reason, Demeter clearly has a nontrivial amount of power over Zeus, and the myth provides an insightful view of how the various power dynamics of the gods happen in action. 
 

\end{document}
