\documentclass[letterpaper, reqno,11pt]{article}
\usepackage[margin=1.0in]{geometry}
\usepackage{color,latexsym,amsmath,amssymb,graphicx, float}
\usepackage{hyperref}

\hypersetup{
colorlinks=true,
linkcolor=magenta,
filecolor=magenta,
urlcolor=cyan,
}
\usepackage{setspace}
\doublespacing

\graphicspath{ {images/} }

\begin{document}
\pagenumbering{arabic}
\title{AMNE 151 Close Reading 2}
\date{27/01/23}
\author{Xander Naumenko}
\maketitle

{\bf Question:} How does this text understand political power? What does it tell us about who is considered the powerless or disenfranchised?  

\medskip

Hesiod's \emph{Works and Days} provides insight into the relative political power between the Olympian ruling class and mortals. More specifically, it highlights how from the perspective of the Olympians, mortals in some context aren't deserving of agency or moral consideration outside of the schemes of their many plans despite humans' fundamental similarity to the gods. 

At the time that the story takes place the Olympians are at their most powerful, while mortal man is described as floundering without the gift of fire. When Prometheus does finally take fire to mortals and anger Zeus, Zeus's response to Prometheus is clearly angry and spiteful. Especially when Zeus says to Prometheus ``you rejoices at your theft of fire and at the trick you played on me'' (\emph{Works and Days}, line 56) and explains how he will ruin Prometheus's creation, it's plain to see that Zeus's motivation is driven primarily by a personal vendetta with Prometheus rather than a desire to maintain natural order or whatever else. But what this reveals is that the actual humans themselves are completely objectified when it comes to conflicts between gods. To them humans are merely a prized creation of Prometheus to be used against him, not a distinct class of people with goals and other consideration. In many ways this is similar to power struggles and colonialism in a less mythological setting. When rulers are in personal conflict with one another it's easy for them to consider massive swathes of people under their rule, perhaps on another continent, as merely another political chip to trade in regardless of their welfare. Just like how Zeus doesn't hesitate to punish all of humanity to get back at Prometheus, awful deeds can happen when there's such an imbalance of power between peoples.  

What's especially ironic about how Zeus doesn't hesitate to punish a massive group due to the actions of another is how similar they are to the gods. In many ways humans were made in the image of the gods, which is especially clear upon the creation of Pandora. The text directly states that Hephaestus tried to "make it look like the immortal goddesses" (\emph{Works and Days}, line 64), and many of the gifts she subsequently receives are directly tied to the strengths and interests of the various goddesses. Despite how similar humans are to the gods though at a completely fundamental level, it still doesn't prevent any of the dehumanizing or punishment that occurs. Our current moral standards of today aren't necessarily compatible with those of ancient Greece, but it's still interesting that so little consideration for the harm to an entire race was given by Zeus upon his punishment of Prometheus. 


\end{document}
