\documentclass[letterpaper, reqno,11pt]{article}
\usepackage[margin=1.0in]{geometry}
\usepackage{color,latexsym,amsmath,amssymb,graphicx, float}
\usepackage{hyperref}

\usepackage[
backend=biber,
style=alphabetic,
sorting=ynt
]{biblatex-chicago}
\addbibresource{sources.bib}

\hypersetup{
colorlinks=true,
linkcolor=magenta,
filecolor=magenta,
urlcolor=cyan,
}
\usepackage{setspace}
\doublespacing

\graphicspath{ {images/} }

\begin{document}
\pagenumbering{arabic}
\title{AMNE 151 Close Reading 2}
\date{February 3rd, 2023}
\author{Xander Naumenko}
\maketitle

{\bf Prompt:} 2. What does the myth of Pandora tell us about how the ancient Greeks thought about sex/gender and about their views of ancient Greek women? Why is the myth of Prometheus tied to the myth of Pandora?  

\medskip

The myth of Pandora shows how the ancient Greek assign Pandora, and by extension women as a whole, a distinct lack of agency in her perpetrated faults, especially compared to Prometheus in his closely related myth. 

In Hesiod's \emph{Works and Days} \autocite{works}, it is made clear that Pandora is created specifically to release her slew of evils upon the world. Hesiod describes the process that Zeus took preparing her for her misdeed, and when Zeus gloats over Prometheus he is very upfront about how he will explicitly create her for the task. And so while the myth implicitly blames the introduction of Pandora for the many hardships and sorrows that men must face, it's not as if she had practically any choice in her role. Given that Pandora is the first human woman, she seems to be emblematic of the Greek view on the female gender as a whole. In this way the myth of Pandora could be read to show how the ancient Greeks thought women were simply created intrinsically to be a burden on mankind, it's not even their actions or choices that deserve ire. 

This is especially contrasting to the Prometheus, the reason for Pandora's creation. In Aeschylus's \emph{Prometheus Bound} \autocite{bound}, Prometheus is extremely accepting of the fact that he had full knowledge of his actions' consequences and made the choices he did regardless. When he speaks to Chorus about how he stole fire for humanity and how as a result he was chained to the rocks, he says ``I knew all along what would happen. I did it anyway– I don’t deny it.'' (Line 265) Unlike Pandora, Prometheus was personally responsible for his choices that had a huge impact on humanity. This is fitting given Pandora and Prometheus in some ways play opposite roles in the same story; Prometheus is the punished and Pandora is the punishment. In this lens Prometheus can be seen as representative of the Greek view on men where they were to make active choices to better society despite potential personal punishment, whereas women were seen to be passively nefarious by their very nature. 

\printbibliography

\end{document}
