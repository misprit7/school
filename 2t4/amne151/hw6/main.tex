\documentclass[letterpaper, reqno,11pt]{article}
\usepackage[margin=1.0in]{geometry}
\usepackage{color,latexsym,amsmath,amssymb,graphicx,float,listings,tikz}
\usepackage{hyperref}

\hypersetup{
colorlinks=true,
linkcolor=magenta,
filecolor=magenta,
urlcolor=cyan,
}

\graphicspath{ {images/} }
\usepackage{setspace}
\doublespacing

\begin{document}
\pagenumbering{arabic}
\title{ANNE Close Reading 6}
\date{10/02/23}
\author{Xander Naumenko}
\maketitle

{\bf Question:} How does this text understand colonialism or imperial conquest? What does it tell us about colonialized or conquered people? 

Homer's {\em Iliad}, Book 5 shows how different the ancient Greek view of war and conquest was than how society views it today. In it, Athena is shown to rally the Greek troops before an encounter with the Trojans. What's most striking about it is her reproach for anyone deemed hesitant to fight and conquer. She criticizes Tydeus' son by telling him ``I shield you from harm, ready to urge you on against the Trojans, yet you seem too tired to attack again or fear robs you of your strength. If that is so, then you are no child of Tydeus, Oeneus' warlike son!'' The strong social pressure to be brave and fight makes sense in social context of the time, when wars and smaller scale skirmishes happened regularly. It makes sense then that their mythology would reflect this. Athena's scolding is in many ways emblematic of the social landscape of the time, where the idea of strongly avoiding armed conflict where possible was deemed cowardly. 

It's also important to analyze Athena's rallying of the Greek soldiers in the social and political context that it occurred in. Athena is supporting the Greeks on a campaign to actively seek out and conquer troy. While there was a nominal reason for the incitement of the war (Paris's choice of Aphrodite over Athena and Hera), the war drags out over a decade and it resembles much like a war of conquest to seize more land. In the context of war today, that's a huge difference; simply compare the public support for current day Ukraine compared to conscription for the Vietnam war. In Iliad, there doesn't seem to be any difference. Whether attacking a foreign power or defending one's homeland, the expectation is that if one doesn't fight then they are a coward. 

\end{document}
