\documentclass[letterpaper, reqno,11pt]{article}
\usepackage[margin=1.0in]{geometry}
\usepackage{color,latexsym,amsmath,amssymb,graphicx,float,listings,tikz}
\usepackage{hyperref}

\hypersetup{
colorlinks=true,
linkcolor=magenta,
filecolor=magenta,
urlcolor=cyan,
}

\graphicspath{ {images/} }
\usepackage{setspace}
\doublespacing

\begin{document}
\pagenumbering{arabic}
\title{ANNE Close Reading 10}
\date{10/02/23}
\author{Xander Naumenko}
\maketitle

{\bf Question:} How does this text understand political power? What does it tell us about who is considered the powerless or disenfranchised?

\medskip

The character of Medea as described in Ovid's {\em Heroides 12, Medea to Jason} reflects many of the implicit relative powerlessness of women in ancient Greece.

\medskip

Medea contributes tremendously to the lasting success of Jason throughout his challenges at great personal cost. When Aeetes was pursuing the Argonauts it was Medea that murdered her brother Absyrtus, and while this was clearly not the ethical thing to do even by Greek standards (as shown by Zeus's disapproval), it does show Medea's near complete devotion to Jason and is successful in delaying their pursuers. Not only that, but one must consider the emotional cost on Medea; she completely betrays and abandons her family. While lamenting what she will do in the wake of Jason's unfaithfulness, she says ``my father is betrayed, kingdom and country forsaken, for which, it is right, my reward's to suffer exile'' (there are no line numbers in this text, around halfway through). This shows that Medea understood that her actions will yield devastating long term consequences, but believed that her relationship with Jason was worth it. 

\medskip

Medea is also very aware of the importance of her contributions. When discussing why she deserves to be Jason, she considers the fruits of their collective exploits as the dowry suitable for their continued relationship. She says ``My dowry’s that golden ram known by its thick fleece, that you’d deny me if I said to you: `Return it'. My dowry is your safety: my dowry’s the youth of Greece.'' (second paragraph from the end) This quote emphasizes how Medea is under no illusions about her importance in Jason's exploits. She calls him out on his reliance to her, with the desperate hope that he recognizes her past faithfulness as worthy of repayment in kind.

\medskip

Despite the clear sacrifices Medea made for Jason, the premise of the letter as a whole is still slanted upon the assumption that it is Medea who must appeal to Jason as the authority in their relationship. She appeals to him by saying ``I beg you, by the gods, by the light of the Sun, my grandfather’s fire, by my kindness to you, and by our two children, our pledges, return to the bed for which I, insanely, abandoned so many things!'' (fourth paragraph from the end) She was taken by surprised as described earlier in the text by Jason's decision, and it is implicit that she must be the one to lay herself bare begging Jason to retake her despite committing no faults (well, relationship faults at least, she murdered a bunch of people). This glaring power gradient seems to be the implicit assumption in many of the interactions between genders in the primary texts we've read thus far which likely reflects the underlying ancient Greek views on the matter.

\end{document}
