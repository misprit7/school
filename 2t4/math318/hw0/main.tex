\documentclass[letterpaper, reqno,11pt]{article}
\usepackage[margin=1.0in]{geometry}
\usepackage{color,latexsym,amsmath,amssymb,graphicx, float}
\usepackage{hyperref}

\hypersetup{
colorlinks=true,
linkcolor=magenta,
filecolor=magenta,
urlcolor=cyan,
}

\graphicspath{ {images/} }

\begin{document}
\pagenumbering{arabic}
\title{Math 318 Homework 0}
\date{11/01/23}
\author{Xander Naumenko\ 38198354}
\maketitle

{\noindent\bf Question 1.} See this document.  

{\noindent\bf Question 2a.} The number of arrangements will be the number of ways of arranging all of the letters and then dividing by the arrangements of each shared letter type, as their order doesn't matter: 

\[
    \frac{11!}{5!\cdot 2!\cdot 2!} = 83160
.\]

{\noindent\bf Question 2b.} This is the same as above except without the As since they're forced to be at the start: 

\[
    \frac{6!}{2!2!}=180
.\]

{\noindent\bf Question 2c.} We can consider the grouping of the 5 As as a single letter, since they have to be together. Therefore the number of arrangements using the same logic as before is: 

\[
    \frac{7!}{2!2!}=1260
.\]

{\noindent\bf Question 2d.} To find this, consider first all of the arrangements of the letters excluding {\bf A}s. All of the arrangements without consecutive {\bf A}s can be generated by taking these words and adding the 5 {\bf A}s between the letters, including before and after the word. For example one possible arrangement of the 6 other letters is BRCDBR, and the places for the {\bf A}s is the underlines in "\_B\_R\_C\_D\_B\_R\_". Since there are 7 places to put the 5 {\bf A}s, we multiply that number of arrangements by seven choose 5: 

\[
    \frac{6!}{2!2!}\cdot 7\choose 5 = 3760
.\]

{\noindent\bf Question 3.} See the python file attached. 

\end{document}
