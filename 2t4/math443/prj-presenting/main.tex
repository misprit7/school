\documentclass[letterpaper, reqno,11pt]{article}
\usepackage[margin=1.0in]{geometry}
\usepackage{color,latexsym,amsmath,amssymb,graphicx,float,listings,tikz}
\usepackage{hyperref}

\hypersetup{
colorlinks=true,
linkcolor=magenta,
filecolor=magenta,
urlcolor=cyan,
}

\graphicspath{ {images/} }

\begin{document}
\pagenumbering{arabic}
\title{Math 443 Project: Presenting}
\date{29/03/23}
\author{Xander Naumenko}
\maketitle

{\medskip\noindent\bf Reflection:} For the final presentation I definitely want to simplify the slides a lot, I heavily overestimated how easy it would be to present relatively dense math proofs quickly. Specifically much of the subtler justification how certain parameters were used were definitely too much, and I'll remove it for the final presentation. I also think some more visuals at the beginning of the presentation would have been good, especially about the difference between recognizability and weak reconstructibility. These are super key to understand for the rest of the paper to make sense so I should have spent some more time on them.

{\medskip\noindent\bf Partner's Understanding:} Here is the paragraph that my partner sent me: 

\medskip

Xander’s presentation explored the reconstruction conjecture. The paper revolved around 5 families of graphs differentiated by their diameter, complement, and edge minimalness. One key lemma of the paper is that if a graph G has diameter greater than three, then the diameter of its complement is less than three. This helps relate all graphs within the 5 families. The paper differentiates between recognizability and weakly reconstructible. The paper shows that the family of a graph is recognizable but leave reconstructibility as future research. That is, it remains to show that each graph within a family has a unique deck. 

Xander’s presentation included clear motivations for why we might look at different parameters, such as $pv$ and $pav$, which were outlined in the paper and used to show recognizability.

\medskip

Overall I'm happy with the summary, although I may have overemphasized that lemma he mentions as its a relatively basic result that isn't really a key lemma of the paper or anything. I'm also happy he correctly identified the distinction between recognizable and weak reconstibility with references with the families of graphs, although this is probably since I clarified it after I presented. I suspect many of the details of the $pv$ and $pav$ parameters I presented on weren't entirely clear given how little he mentioned them, but I plan on deemphasizing for the final presentation anyways so I'm not too worried about that.


\end{document}
