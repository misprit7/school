\documentclass[letterpaper, reqno,11pt]{article}
\usepackage[margin=1.0in]{geometry}
\usepackage{color,latexsym,amsmath,amssymb,graphicx,float,listings,tikz}
\usepackage{hyperref}

\hypersetup{
colorlinks=true,
linkcolor=magenta,
filecolor=magenta,
urlcolor=cyan,
}

\graphicspath{ {images/} }

\begin{document}
\pagenumbering{arabic}
\title{Math 443 Project Reflection}
\date{13/04/23}
\author{Xander Naumenko}
\maketitle

The biggest takeaway from the presentations is how difficult it is to present math. Generally I've found it quite easy to wing presentations by just preparing a rough outline of what I wanted to say and making a quick slide deck, but when you're giving a highly technical talk about proofs this is just completely impossible. It's definitely worth taking a lot of time to carefully think over exactly what the best way to convey the ideas of a paper are. Specifically, slides filled with algebra are seem to be reasonable when you're preparing them, but they are completely impossible to parse as a listener. Abstracting over the high level ideas of a proof is far better in that circumstance, in my experience listeners are always open to trust the presenter on the algebra and will take away more from the presentation anyways.

Another big takeaway from the presentations is how hard it is to strike a balance between introductory high level explanations and the actual meat of the proof. For our presentation we leaned heavily towards the introductory side to give listeners hopefully a good baseline level of knowledge and skipped over most of the paper's proof, while some other teams went much more heavily in the other direction. I this balance is really hard to perfect, personally I thought the presentation on dominators/semi-dominators did this best among all of the presentations. 

Finally I also found that diagrams were really the thing that made or broke a presentation. A good diagram was enough to carry a lackluster explanation by itself, and even the most eloquent description of a proof was completely lost without a good diagram. I think this is because diagrams are the only way to convey the intuition behind a proof at least for graph theory, and intuition is the most important part of a proof. In the future I think it makes sense to plan the entire slide deck around a small set of diagrams that convey the right intuition and structure the text around these.


\end{document}
