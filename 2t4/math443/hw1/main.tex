\documentclass[letterpaper, reqno,11pt]{article}
\usepackage[margin=1.0in]{geometry}
\usepackage{color,latexsym,amsmath,amssymb,graphicx,float,tikz}
\usepackage{hyperref}

\usetikzlibrary {graphs}

\hypersetup{
colorlinks=true,
linkcolor=magenta,
filecolor=magenta,
urlcolor=cyan,
}

\graphicspath{ {images/} }

\begin{document}
\pagenumbering{arabic}
\title{Math 443 Homework 1}
\date{18/01/23}
\author{Xander Naumenko}
\maketitle

{\medskip\noindent\bf Question 1a.} Yes, $G$ is bipartite. The two components are the set of vertices containing $P$ and those that don't. This is because the person is always on the paddleboard, so you can't go between two configurations that do or don't have the person on one side. 

{\medskip\noindent\bf Question 1b.} 


\tikz \graph { GCFP -- GF -- GFP -- {F, G} -- {CFP, GCP} -- C -- CP -- emptyset/$\emptyset$ };

{\medskip\noindent\bf Question c.} Yes. 

{\noindent\bf Question 1d.} The solution will look like a path from with one endpoint $\{G, C, F, P\}$ and the other endpoint $\emptyset$. This path represents a sequence of paddleboard trips to solve the problem. 

{\medskip\noindent\bf Question 1e.} From the graph in part b, there are two possible solutions: 
\[
GCFP\to GF\to GFP\to F\to CFP\to C\to CP\to \emptyset
\]
and
\[
GCFP\to GF\to GFP\to G\to GCP\to C\to CP\to \emptyset
.\]

{\medskip\noindent\bf Question 2a.} The statement is false. A counterexample is $i=5, j=8$. Then as shown in the given sub graph, $v_i=v_5$ is created on the 4th iteration whereas $v_j=v_8$ was created on the 3rd iteration. 

{\medskip\noindent\bf Question 2b.} The statement is true. The second part is easy: since $i$ is even it can be represented by $i=2j$, so $\frac{3i+1}{2}=3j+\frac{1}{2}\not\in \mathbb{N}$, whereas each vertex is labeled with a natural number. Thus it is impossible that $v_i$ is adjacent to $v_{\left( 3i+1 \right) /2}$. 

For the first part, note that $v_i$ must have preceding neighbor that generated it (or else it would be either $v_1$ or not be in the graph, and $1$ isn't even). There are only two ways to generate subsequent vertices as described in the problem statement. It will be shown that the second method of generating vertices will never generate even vertices, so the preceding neighbor of $v_i$ is $v_{\frac{i}{2}}$. Let $m\equiv 2\mod 3, m=3n+2$, then $\frac{2i-1}{3}=\frac{6n+3}{3}=2n+1$, which is odd. Thus it can't be that $v_i$ was generated this way, so it must have been generated by the first method which implies that $v_i$ is adjacent to $v_{i/2}$. $\square$

{\medskip\noindent\bf Question 2c.} The statement is true. We will use proof by contradiction, so suppose that $v_i$ is created twice in the graph. Note that both methods of generating new vertices are reversible, i.e. if $v_i\in G, i\neq 1$ then either $v_{\frac{3i+1}{2}}\in G$ or $v_{\frac{i}{2}}\in G$. Since we assume $v_i$ is created twice, it must be that $v_{\frac{3i+1}{2}}\in G$ and $v_{\frac{i}{2}}\in G$. 

Since all vertices in $G$ are indexed by natural numbers as mentioned above, $v_{\frac{i}{2}}\in G \implies i=2j, j\in\mathbb{N}$. Using our results from part b, this means that $v_i$ is not adjacent to $v_{\frac{3i+1}{2}}$, which contradicts the fact that it must be to have fulfilled our original hypothesis. Thus our assumption must have been wrong and $v_i$ isn't generated twice. $\square$

{\medskip\noindent\bf Question 2d.} The statement is true. Again we will use proof by contradiction, so suppose that there exists a finite cycle $C$ in $G$. Let $v_i$ be a youngest vertex in $C$, i.e. a vertex that is generated after or in the same iteration as every other vertex in $C$. Since $C$ is a cycle every vertex within it has exactly two neighbors also in $C$. By part c of this question $v_i$ has only one preceding neighbor, so $v_i$ must generate at least one vertex below it that is in $C$ (more specifically either $v_{2i}\in C$ or $v_{\frac{2i-1}{3}}\in C$). However this contradicts our assumption that $v_i$ is a youngest vertex in $C$, since we've just generated an newer one, so our assumption that cycle exists must be incorrect and there are no cycles in $G$. $\square$

{\medskip\noindent\bf Question 2e.} $\forall i\in\mathbb{N}, v_i\in G$. 

{\medskip\noindent\bf Question 3.} The statement is false, for example consider the graph $G$ with vertex set $V=\{v_1, v_2\}$ and edge set $E=\emptyset$. Let $H\subset G$ have vertex set $\{v_1\}$ and edge set $\emptyset$. Then  $V(H)\neq V(G)$ so $H$ is proper, but $\left| E(H) \right| =\left| E(G) \right| =0$ so the definition isn't equivalent. 

{\medskip\noindent\bf Question 4.} 

\tikzstyle{vertex}=[shape=circle, minimum size=2mm, fill, draw, inner sep=0]

One vertex:

\begin{center}
\begin{tikzpicture}
\draw (0,0)node[vertex]{};
\end{tikzpicture}
\end{center}

Two vertices:

\begin{center}
\begin{tikzpicture}
\draw (0,0)node[vertex]{};
\draw (1,0)node[vertex]{};
\draw (3,0)node[vertex]{};
\draw (4,0)node[vertex]{};
\draw (0,0)node[vertex]{};
\draw (0,0)--(1,0);
\end{tikzpicture}
\end{center}

Three vertices:

\begin{center}
\begin{tikzpicture}
\draw (0,1)node[vertex]{};
\draw (1,1)node[vertex]{};
\draw (1,0)node[vertex]{};

\draw (3,1)node[vertex]{};
\draw (4,1)node[vertex]{};
\draw (4,0)node[vertex]{};

\draw (6,1)node[vertex]{};
\draw (7,1)node[vertex]{};
\draw (7,0)node[vertex]{};

\draw (9,1)node[vertex]{};
\draw (10,1)node[vertex]{};
\draw (10,0)node[vertex]{};

\draw (0,1)--(1,1);
\draw (1,1)--(1,0);
\draw (0,1)--(1,0);

\draw (3,1)--(4,1);
\draw (4,1)--(4,0);

\draw (6,1)--(7,1);

\end{tikzpicture}
\end{center}

Four vertices:

\begin{center}
\begin{tikzpicture}
\draw(0,0) rectangle (1,1);
\draw (0,0)--(1,1);
\draw (0,0)node[vertex]{};
\draw (0,1)node[vertex]{};
\draw (1,0)node[vertex]{};
\draw (1,1)node[vertex]{};

\draw(2,0) rectangle (3,1);
\draw (2,0)node[vertex]{};
\draw (2,1)node[vertex]{};
\draw (3,0)node[vertex]{};
\draw (3,1)node[vertex]{};

\draw (4,0)node[vertex]{};
\draw (4,1)node[vertex]{};
\draw (5,0)node[vertex]{};
\draw (5,1)node[vertex]{};
\draw (4,1)--(5,1);
\draw (4,0)--(4,1);
\draw (4,1)--(5,0);
\draw (5,1)--(5,0);

\draw (6,0)node[vertex]{};
\draw (6,1)node[vertex]{};
\draw (7,0)node[vertex]{};
\draw (7,1)node[vertex]{};
\draw (6,1)--(7,1);
\draw (6,1)--(7,0);
\draw (7,1)--(7,0);

\draw (8,0)node[vertex]{};
\draw (8,1)node[vertex]{};
\draw (9,0)node[vertex]{};
\draw (9,1)node[vertex]{};
\draw (8,1)--(9,1);
\draw (8,1)--(9,0);
\draw (8,1)--(8,0);

\end{tikzpicture}
\end{center}

\begin{center}
\begin{tikzpicture}
\draw (0,0)node[vertex]{};
\draw (0,1)node[vertex]{};
\draw (1,0)node[vertex]{};
\draw (1,1)node[vertex]{};
\draw (0,0)--(0,1);
\draw (0,0)--(0,1);
\draw (0,1)--(1,1);
\draw (1,0)--(1,1);

\draw (2,0)node[vertex]{};
\draw (2,1)node[vertex]{};
\draw (3,0)node[vertex]{};
\draw (3,1)node[vertex]{};
\draw (2,0)--(2,1);
\draw (2,1)--(3,1);

\draw (4,0)node[vertex]{};
\draw (4,1)node[vertex]{};
\draw (5,0)node[vertex]{};
\draw (5,1)node[vertex]{};
\draw (4,0)--(4,1);
\draw (5,0)--(5,1);

\draw (6,0)node[vertex]{};
\draw (6,1)node[vertex]{};
\draw (7,0)node[vertex]{};
\draw (7,1)node[vertex]{};
\draw (6,0)--(6,1);

\draw (8,0)node[vertex]{};
\draw (8,1)node[vertex]{};
\draw (9,0)node[vertex]{};
\draw (9,1)node[vertex]{};

\end{tikzpicture}
\end{center}

{\medskip\noindent\bf Question 5a.} The statement is true. Note that $|G|\geq 3$ and $u,v$ fulfill the requirements for the theorem stated at the beginning of the question, so $G$ is connected by this theorem (i.e. the statement to be proved is strictly weaker than the theorem proved in class). $\square$

{\medskip\noindent\bf Question 5b.} The statement if false. Let $G$ be a graph with vertex set $\{v_1, v_2, v_3, v_4\}$ and edge set $\{\{v_1, v_2\}, \{v_2, v_3\} , \{v_3, v_4\} \} $. Then letting $u=v_1$ and $v=v_4$ work for $u,v$, but $w$ must then be either $v_2$ or $v_3$ which separates either $v_1$ or $v_4$ from the rest of the graph, so it won't be connected. 

{\medskip\noindent\bf Question 6.} The statement is true. The proof will be constructive, so let $G$ be a graph with a shorted cycle $C=\{v_1, v_2, \ldots, v_{| |C| |}\} $ of length $||C||= g(G)$ and we will find two vertices of sufficient distance. Let $v_i$ be an arbitrary vertex of $C$, and $v_j\in C$ be a vertex in $C$ such that $j\equiv 2i\mod | |C| |$. Consider the paths $P_1=| |v_i v_{i+1}\ldots v_{j-1}v_j| |$ and $P_2=| |v_j v_{j+1}\ldots v_{i-1}v_i| |$. $P_1P_2=C$ so they are related by $| |P_1| |+| |P_2| |=| |C| |\implies | |P_1| | \geq\frac{| |C| |-1}{2}$ and $| |P_2| | \geq\frac{| |C| |-1}{2}$. Without loss of generality assume that $| |P_1| | \leq | |P_2| |$.  

Note that it must be that $P_1$ is a shortest path between $v_i$ and $v_j$. To see why, assume that it wasn't, i.e. assume there exists a path $P'$ with endpoints $v_i, v_j$ with $| |P_1| |> | |P'| |$ ($P'\neq P_2$ since $| |P_1| |\leq | |P_2| |$. Then the walk $P'P_2$ is a closed walk, and theorem 1.6 tells us that there exists a cycle $C'$ s.t. $| |C'| |\leq | |P'P_2| |<| |P_1P_2| |=| |C| |$. However by assumption $C$ was the smallest cycle in $G$, so therefore $P_1$ must have been the shortest path between $v_iv_j$. Thus the distance between $v_i,v_j$ is $| |P_1| |\geq \frac{| |C| |-1}{2}$, so the diameter $G$ is at least $diam(G)\geq \frac{| |C| |-1}{2}\implies g(G)\leq 2diam(G)+1$, so we're done. $\square$

{\medskip\noindent\bf Question 7a.} Let $P'=P-\{a,b\} $ and $Q'=Q-\{a,b\} $. Then $V(P')\cap V(Q')=\emptyset$, and consider the walk $w=aP'bQ'a$. $w$ is closed since it starts and ends on the same vertex, and note that it contains no repeated vertices since $V(P')\cap V(Q')=\emptyset$. Thus $w$ is in fact a cycle, and has order $1 + |P'| + 1 + |Q'|=2 + |P|-2+|Q'|\geq |P| $. The length of a cycle is its order, so we're done. $\square$

{\medskip\noindent\bf Question 7b.} If $| |C| |> \sqrt{k} $ then we're done since we can just choose $C$ as our cycle, so assume that $| |C| |< \sqrt{k} $. Consider the following method of generating subpaths of $P$: start by letting $v_i$ to be the first vertex in $P$ s.t. $v_i\in V(C)$ (this is always possible since $v_k\in V(C)$), so $P=v_1v_2\ldots v_i\ldots v_k$, and let $P'=v_1v_2\ldots v_{i-1}v_{i}$ and $P''=v_{i}v_{i+1}\ldots v_{k-1}\ldots v_{k}$. Repeat this process except with $P''$ as $P$, this is guaranteed to terminate since $P$ has finite length and each iteration reduces the length of the remaining path. Let $P_1, P_2, \ldots, P_n$ to be the list of $P'$ generated this way in order. Because this process repeatedly splits $P$ into subpaths, $P_1P_2\ldots P_n=P$.

Since there are no repeated vertices in $P$ since it's a path and each $P_i$ contains a new vertex from $C$ not contained in $P_j$ for all $j<i$, $n\leq |C|=| |C| |< \sqrt{k} $. We also know that $| |P| |=|| P_1| |+| |P_2| |+\ldots+| |P_n| |=| |P| |=k$ because each $P_i$ was generated by splitting $P$ into subpaths. Since we have $n$ paths to divide a total length of $k$, at least one path $P^*\in \{P_1, P_2, \ldots, P_n\} $ must have length $||P^*| |\geq\frac{k}{n}> \frac{k}{\sqrt{k} }=\sqrt{k} $. The endpoints of $P^*=a\ldots b$ are on a cycle, so there exists a path $Q$ from $a$ to $b$ s.t. $Q\subset C$. By construction $P^*\cap Q=\{a,b\}$, so applying what was proven in part a we can conclude there exists a cycle $C'$ in $G$ of length at least 
\[
| |C'| |\geq |P^*|> \sqrt{k}. \ \square 
\]

{\medskip\noindent\bf Question 8.} Consider a longest path $P=v_1v_2\ldots v_n$ in $G$. The neighbors of $v_1$ must be in $P$, or else it wouldn't be a longest path since you could append them to $P$ to make a longer one. Since $\delta(G)\geq 3$, there are at least $3$ neighbors, $v_2, v_i, v_j\in P$. WLOG assume $i<j$. Now consider the cycle $C=v_1v_2\ldots v_i\ldots v_jv_1$ and the edge $e=v_1v_i$. $e$ added to $C$ forms a chorded cycle, so we're done. $\square$

\end{document}
