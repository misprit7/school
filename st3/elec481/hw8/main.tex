\documentclass[letterpaper, reqno,11pt]{article}
\usepackage[margin=1.0in]{geometry}
\usepackage{color,latexsym,amsmath,amssymb,graphicx, float}
\usepackage{hyperref}

\hypersetup{
colorlinks=true,
linkcolor=magenta,
filecolor=magenta,
urlcolor=cyan,
}

\graphicspath{ {images/} }

\begin{document}
\pagenumbering{arabic}
\title{ELEC 481 Homework 8}
\date{16/06/22}
\author{Xander Naumenko}
\maketitle

{\noindent\bf Question 1.} 

Cost of materials: Direct. 

Interest payments: Indirect. 

Machine depreciation: Indirect. 

Product handling: Direct. 

Machine operator wages: Direct. 

Utility costs: Indirect. 

Support staff salaries: Indirect. 

Marketing cost: Indirect. 

Storing cost: Indirect. 

Insurance costs: Indirect. 

Engineering drawings: Direct. 

Machine operator overtime expenses: Direct. 

Tooling costs: Direct. 

{\noindent\bf Question 2.} One option would be government bonds. According to the bank of Canada website, this would result in an interest rate of approximately 3.15\% with practically no risk. Another option would be to invest in the stock market, for example through the S&P500. The average annual rate of return over the last 5 years has been approximately 17\%, so on the basis of the past this seems like a reasonable baseline to use. Finally you could also invest in individual stocks, with the hopes that the one you choose goes up. This option has by far the most risk as you are not insulated from how that company does, but also has the highest upside. Given that I don't consider myself a particularly great investor and would effectively be picking at random, I would expect approximately the same rate of return as just investing in the market as a whole, so around 17\% (the efficient market guarantees that guessing randomly can't be too much worse than the market rate). 

As for the minimum attractive rate of return, it's difficult to put a concrete number on it but I would say approximately 15\%. Given that my earnings will presumably go up with time money now is certainly worth more now than in the future to spend on my enjoyment. 

{\noindent\bf Question 3a.} The depreciation value in year 5 would be 
\[
CCP=P (1-\frac{d}{2})(1-d)^{n-2}d=400000 (1-0.15)0.7^{3}0.3=\$34986
.\]

{\noindent\bf Question 3b.} In this case it's just: 
\[
d_t=\frac{400000-75000}{7}=\$46429
.\]

{\noindent\bf Question 3c.} 
\[
d_t=(7-5+1)\cdot \frac{2}{7\cdot 8}(400000-75000)=\$34821
.\]

{\noindent\bf Question 3d.} 
\[
d_5=1.5/7 \cdot 400000 \left( 1-\frac{1.5}{7} \right)^{4}=\$32667
.\]

{\noindent\bf Question 4a.} Book value: 
\[
BV=80000 + \frac{2}{8}\left( 600000-80000 \right) =\$210000
.\]

Thus the difference between the two prices is \$30000. 

{\noindent\bf Question 4b.} Recaptured depreciation happen because it sold for more than its book value. 

{\noindent\bf Question 4c.} The firm would owe $30000\cdot 0.33=\$9900$ since that's the amount that was earned more than accounted for by depreciation. 


\end{document}
