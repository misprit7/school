\documentclass[letterpaper, reqno,11pt]{article}
\usepackage[margin=1.0in]{geometry}
\usepackage{color,latexsym,amsmath,amssymb,graphicx,float,listings,tikz}
\usepackage{hyperref}

\hypersetup{
colorlinks=true,
linkcolor=magenta,
filecolor=magenta,
urlcolor=cyan,
}

\graphicspath{ {images/} }

\begin{document}
\pagenumbering{arabic}
\title{ELEC 302 Lab 1}
\date{05/02/24}
\author{Xander Naumenko}
\maketitle

\section{Task 1}

{\medskip\noindent\bf Part 1.} 

{\medskip\noindent\bf Part 2.} 

For the capacitor, we can use the formula given in class to calculate the value:
\[
C = \frac{V_p-V_d}{fR_LV_r}=\frac{16.8-0.7}{60\cdot 1000\cdot 1}=268.3\mu\text{F}
.\]
Since we don't have any capacitor values close to that, we instead used one 220$\mu$F and two $33\mu$F capacitors in parallel which has an equivalent capacitance of $286\mu$F.

\section{Task 2}

{\medskip\noindent\bf Part 1.} 

For the capacitor using the equation:
\[
C = \frac{V_p-2V_d}{2fR_LV_r}=\frac{33.2-1.4}{2\cdot 60\cdot 1000\cdot 1}=265\mu\text{F}
.\]
Again we used the same capacitor configuration as before of $286\mu$F since we didn't have a single capacitor of the correct value.

{\medskip\noindent\bf Part 2.} 

Again, the capacitor comes from the equation:
\[
C = \frac{V_p-V_d}{2fR_LV_r}=\frac{16.8-0.7}{2\cdot 60\cdot 1000\cdot 1}=134\mu\text{F}
.\]
Three 33$\mu$F capacitors were used in parallel to simulate this.

\end{document}
