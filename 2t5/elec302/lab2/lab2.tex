\documentclass[letterpaper, reqno,11pt]{article}
\usepackage[margin=1.0in]{geometry}
\usepackage{color,latexsym,amsmath,amssymb,graphicx,float,listings,tikz}
\usepackage{hyperref}

\hypersetup{
colorlinks=true,
linkcolor=magenta,
filecolor=magenta,
urlcolor=cyan,
}

\graphicspath{ {images/} }

\begin{document}
\pagenumbering{arabic}
\title{ELEC 302 Lab 2}
\date{04/03/24}
\author{Xander Naumenko}
\maketitle

{\medskip\noindent\bf Question 1.1.} The equations, derived in class [TODO: derive]:
\[
v_{+}= V_{TL}=-L_+ \frac{R_1}{R_2}, V_{TH}= -L_- \frac{R_1}{R_2}
.\]

{\medskip\noindent\bf Question 1.2.} 

\[
\frac{R_2}{R_1}=\frac{L_+}{3}=\frac{13}{3}\approx 4.67
.\]
Thus as  good approxmation we used a $4.7\text{k}\Omega$ and a $1\text{k}\Omega$ resistor for a ratio of $4.7$.

{\medskip\noindent\bf Question 1.4.} Expermintally, $V_{in}=6.5$V

{\medskip\noindent\bf Question 2.1.} From the prelab, $T=RC\text{ln} \frac{V_+-V_-}{V_{ref}}=1.4$

{\medskip\noindent\bf Question 2.2.} Measured: $V_{ref}=-2.75$

\end{document}
