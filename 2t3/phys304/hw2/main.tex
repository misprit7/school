\documentclass[letterpaper, reqno,11pt]{article}
\usepackage[margin=1.0in]{geometry}
\usepackage{color,latexsym,amsmath,amssymb,graphicx, float}
\usepackage{hyperref}

\hypersetup{
colorlinks=true,
linkcolor=magenta,
filecolor=magenta,
urlcolor=cyan,
}

\graphicspath{ {images/} }

\begin{document}
\pagenumbering{arabic}
\title{PHYS 304 Homework 2}
\date{26/01/22}
\author{Xander Naumenko}
\maketitle

{\noindent\bf Question 1a.} To normalize, integrate: 

\[
1=\int_{-\infty}^{\infty}\psi^*\psi dx=\int_{-a}^{a}A^2(a^2-x^2)^2dx=\int_{-a}^aA^2(x^{4}-2a^2x^2+a^{4})dx=A^2\left( \frac{2}{5}a^{5}-\frac{4}{3}a^{5}+2a^{5} \right) =A^2 \frac{16}{15}a^{5}
\]
\[
\implies A=\frac{\sqrt{15}}{4a^{5}}
.\]

{\noindent\bf Question 1b.} To find the average value, apply the $ \left<x \right>$ operator: 

\[
\int_{-\infty}^{\infty}\psi^* x\psi dx=\int_{-a}^{a}A^2(a^2-x^2)^2 xdx=\int_{-a}^aA^2(x^{5}-2a^2x^3+a^{4}x)dx
.\]
Because all the $x$ terms are odd and we're integrating around $0$, the integral goes to zero and the expectation value is $ \left<x \right>=0$. 

{\noindent\bf Question 1c.} To calculate the momentum we use the momentum operator: 
\[
\left<p \right>=-i\overline{h}\int_{-\infty}^{\infty}\psi^*\frac{\partial\psi}{\partial x}dx=-i\overline{h}\int_{-a}^a A^2(a^2-x^2)2xdx
.\]
Because all the terms in the integral are once again odd, the integral goes to zero so $ \left<p \right>=0$. 

{\noindent\bf Question 1d.} Taking the integral: 
\[
\int_{-\infty}^{\infty}\psi^* x^2\psi dx=\int_{-a}^{a}A^2(x^{6}-2a^2x^{4}+a^{4}x^2)dx=A^2a^{7}\left( \frac{2}{7}-\frac{4}{5}+\frac{2}{3}\right)=A^2a^{7}\frac{16}{105}=\frac{1}{7}a^{2}
.\]

{\noindent\bf Question 1e.} Again taking the integral: 
\[
\left<p \right>=-\overline{h}^2\int_{-\infty}^{\infty}\psi^* \frac{\partial^2}{\partial x^2}\psi dx=\overline{h}^2\int_{-a}^{a}2A^2(a^2-x^2)dx=2\overline{h}^2A^2a^{3}\left( 2-\frac{2}{3} \right) =\frac{8}{3}A^2a^{3}\overline{h}^2=\frac{5}{2}a^{-2}\overline{h}^2
.\]

{\noindent\bf Question 1f.} The uncertainty is the standard deviations, so 
\[
\sigma_x=\sqrt{\left<x^2 \right>-\left<x \right>^2}=\frac{1}{\sqrt{7} }a
.\]

{\noindent\bf Question 1g.} Same as last question: 
\[
\sigma_y=\sqrt{\left<p^2 \right>-\left<p \right>^2}=\frac{\sqrt{5} }{\sqrt{2} }a^{-1}\overline{h}
.\]

{\noindent\bf Question 1h.} As can be seen the results confirm the uncertainty principle: 
\[
\sigma_x\sigma_y=\sqrt{\frac{5}{14}} \overline{h}\geq \frac{1}{2}\overline{h}
.\]

{\noindent\bf Question 2a.} Setting the wavelength equal to the characteristic size of the system, we can solve: 

\[
d=0.3nm=\lambda=\frac{h}{\sqrt{3mk_bT} }\implies T=\frac{h^2}{3d^2mk_b}=129342K
.\]
Thus for any reasonable temperatures they act quantum mechanically. For the nucleus it is the exact same calculation except with a different mass: 
\[
T=\frac{h^2}{3d^2mk_b}=2.5K
.\]
Thus the nucleus generally behaves classically. 

{\noindent\bf Question 2b.} First we find $d$. To do so assume that the gas is spread out in a lattice structure, with density of $\frac{N}{V}=\frac{P}{k_bT}$. Thus the intermolecular spacing is $ \sqrt[3]{\frac{V}{N}}=\left( \frac{k_bT}{P} \right)^{1 /3}$. Putting this into our expression for $T $: 
\[
T=\frac{h^2}{3d^2mk_b}=\frac{h^2P^{2 /3}}{3mT^{2/3}k_b^{5 /3}}\implies T=\left( \frac{1}{k_b} \right) \left( \frac{h^2}{3m} \right)^{\frac{3}{5}}P^{\frac{2}{5}} 
\]
as expected. Putting the numbers in, for hydrogen the requisite temperature is 
\[
T=2.937K
.\]
For Helium in the atmosphere, we can solve for what the wavelength:
\[
\lambda=\frac{h}{\sqrt{3mk_bT} }=1.45\cdot 10^{-9}m
.\]
Since this is clearly much smaller than 1cm, the Hydrogen acts classically. 

{\noindent\bf Question 3.} As the hint suggests, rewrite equation 2.5 as 
\[
\frac{d^2\psi}{dx^2}=\frac{2m}{\overline{h}^2}\left( V(x)-E \right) \psi
.\]
Consider the case where $E$ never exceeds the minimum value of $V$, and we will show that this is impossible. In this case then we have that $V(x)-E>0\forall x$. Then from the differential equation we just wrote out this means that the second derivative and the function have the same sign for all $x$. 

Next note that for any $\psi$ that satisfies our differential equation, we can assume that $\psi(0)\geq 0$. This is because if this wasn't true, we could consider the new function $\psi^\prime=-\psi$ which fulfills the differential equation and doesn't affect the square integrability of $\psi$. Similarly, we can assume that $\frac{d\psi}{dx}(0)\geq0$, since if this wasn't the case then we could consider $\psi^\prime(x)=\psi(-x)$ which would also fulfill the differential equation and have no effect on the integrability. Let $\frac{d\psi}{dx}(0)=m$. Since $\frac{d^2\psi}{dx^2}>0$ as long as $\psi>0$, $\psi(x)\geq \psi(0)+mx\forall x\geq0$. This is linear function which clearly doesn't converge even in the limit, which means that $\psi$ is not square integrable. 

{\noindent\bf Question 4.} Take the time derivative inside the integral: 

\[
\frac{d}{dt}\int_{-\infty}^{\infty}\Psi_1^*\Psi_2 dx=\int_{-\infty}^{\infty}\left(\Psi_1^* \frac{\partial \Psi_2}{\partial t} + \frac{\partial \Psi_1^*}{\partial t} \Psi_2\right) dx
.\]
The Schr\"odinger equation, after being manipulated is:
\[
\frac{\partial \Psi}{\partial t} =\frac{i\overline{h}}{2m} \frac{\partial^2\Psi}{\partial x^2} -\frac{i}{\overline{h}}V\Psi
.\]
Plugging this into the first equation and noticing that the potential terms drop out due to the taking of the conjugate, we get 
\[
\frac{d}{dt}\int_{-\infty}^{\infty}\Psi_1^*\Psi_2 dx=\int_{-\infty}^{\infty}\frac{i\overline{h}}{2m}\left( \Psi_1^* \frac{\partial^2\Psi_2}{\partial x^2}-\frac{\partial ^2\Psi_1^*}{\partial x^2}\Psi_2 \right)=\frac{i\overline{h}}{2m}\left( \Psi_1^*\frac{\partial\Psi_2}{\partial x}-\frac{\partial \Psi_1^*}{\partial x}\Psi_2 \right) \bigg|_{-\infty}^\infty
.\]
Since the wavefunctions are presumably normalizable given that they are solutions to the Schr\"odinger, both $\Psi_1, \Psi_2$ must go to zero at infinity. Thus the integral goes to zero and we have 
\[
\frac{d}{dt}\int_{-\infty}^{\infty}\Psi_1^*\Psi_2 dx=0
\]
as required. 

\end{document}
