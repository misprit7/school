\documentclass[letterpaper, reqno,11pt]{article}
\usepackage[margin=1.0in]{geometry}
\usepackage{color,latexsym,amsmath,amssymb,graphicx, float, braket}
\usepackage{hyperref}

\hypersetup{
colorlinks=true,
linkcolor=magenta,
filecolor=magenta,
urlcolor=cyan,
}

\graphicspath{ {images/} }

\begin{document}
\pagenumbering{arabic}
\title{PHYS 304 Homework 6}
\date{09/03/22}
\author{Xander Naumenko}
\maketitle

{\noindent\bf Question 1a.} From the definition of operators: 
\[
\int \psi^* x\psi dx=\int(x\psi)^*\psi dx\implies x^\dagger=x
.\]
\[
\int \psi^* i\psi dx=\int(-i\psi)^*\psi dx\implies i^\dagger=-i
.\]
\[
\int \psi^* \frac{d}{dx}\psi dx=\psi\psi^*\bigg|_{-\infty}^{\infty}-\int(\frac{d}{dx}\psi)^*\psi dx=\int(-\frac{d}{dx}\psi)^*\psi dx\implies \left( \frac{d}{dx} \right) ^\dagger=-\frac{d}{dx}
.\]

{\noindent\bf Question 1b.} Expanding: 

\[
\left<f\mid \left( \hat{Q}\hat{R} \right)^{\dagger}g  \right>=\left<\hat{Q}\hat{R}f\mid g \right>=\left<\hat{R} f\mid \hat{Q}^{\dagger}g \right>=\left<f\mid \hat{R}^{\dagger}\hat{Q}^{\dagger}g \right>\implies\left( \hat{Q}\hat{R} \right) ^{\dagger}=\hat{R}^{\dagger}\hat{Q}^{\dagger}
.\]
\[
\left<f\mid \left( \hat{Q}+\hat{R} \right) ^{\dagger}g \right>=\left<\hat{Q}f\mid g \right>+\left<\hat{R}f\mid g \right>=\left<f\mid \left( \hat{Q}^{\dagger} +\hat{R}^{\dagger}\right) g \right>\implies \left( \hat{Q} +\hat{R}\right)^{\dagger}=\hat{Q}^{\dagger}+\hat{R}^{\dagger}
.\]
\[
\left<f\mid \left( c\hat{Q} \right) ^{\dagger}g \right>=\left<c\hat{Q} f\mid g \right>=\left<f\mid c^*\hat{Q}g \right>\implies \left( c\hat{Q} \right) ^{\dagger}=c^*\hat{Q}^{\dagger}
.\]

{\noindent\bf Question 2.} Note that from what we derived in the textbook, the space dependent wave function for the ground state of the harmonic oscillator is 
\[
\psi(x)=\left( \frac{m\omega}{\pi \hbar} \right)^{1 /4}e^{\frac{m\omega}{2\hbar}x^2}e^{-i\omega t /2}
.\]
Plugging this into equation 3.54: 
\[
\Psi(p, t)=\frac{1}{\sqrt{2\pi\hbar} }e^{-i\omega t /2}\int_{-\infty}^{\infty}e^{-ipx /\hbar}\left( \frac{m\omega}{\pi \hbar} \right)^{1 /4}e^{\frac{m\omega}{2\hbar}x^2}
.\]
Using an integral calculator we find the final equation to be (we could also have completed the square then integrated): 
\[
\Phi(p, t)=\frac{\exp\left( -\frac{p^2}{2m\omega\hbar}- \frac{i\omega t}{2} \right) }{\left( \pi \omega m \hbar \right)^{1 /4} }
.\]

{\noindent\bf Question 3.} Expanding out the definition of the momentum space wave function: 
\[
\int \Phi^* \left( i\hbar \frac{\partial}{\partial p} \right) \Phi dp=\frac{1}{\sqrt{2\pi \hbar} }\int \left(\int e^{ipx /\hbar}\Psi^*(x, t)dx\right)\left( \int -i \hbar \frac{ix}{\hbar} e^{-ipx \hbar}\Psi(x, t)dx \right) dp
.\]
\[
=\frac{1}{\sqrt{2\pi \hbar} }\int \left(\int e^{ipx /\hbar}\Psi^*(x, t)dx\right)\left( \int x e^{-ipx \hbar}\Psi(x, t)dx \right) dp
.\]
This can be simplified as the hint suggests by combining the integrals and using the definition of delta functions, giving us: 
\[
=\int \Psi^*(x, t) x \Psi(x, t)dx=\left<x \right>
.\]

{\noindent\bf Question 4.} Because there are two basis elements, there are two eigenstates/eigenvalues. To find them we must solve: 
\[
\hat{H}\ket{h}=h\ket{h}, \ket h=a\ket 1+b\ket 2
.\]
This is equivalent to the following set of linear equations: 
\[
    \epsilon\begin{pmatrix} 1&1\\1&-1 \end{pmatrix}\begin{pmatrix} a\\b \end{pmatrix} =h\begin{pmatrix} a\\b \end{pmatrix} 
.\]
\[
\implies \begin{cases}
    h=-\sqrt{2}\epsilon \implies \ket h=\left( 1-\sqrt{2}  \right) \ket 1+\ket 2\\
    h=\sqrt{2}\epsilon \implies \ket h=\left( 1+\sqrt{2}  \right) \ket 1+\ket 2
\end{cases}
.\]
As already indicated the matrix representing this set of equations is: 
\[
\epsilon \begin{pmatrix} 1&1\\1&-1 \end{pmatrix}
.\]

{\noindent\bf Question 5a.} The measurement collapses the state of the system, so the state of the system is then $\psi_1$. 

 {\noindent\bf Question 5b.} The two possible results are $\psi_1$ or $\psi_2$ with eigenvalues $b_1$ and $b_2$ respectively. The state is currently $\psi_1=\left( 3\psi_1+4\psi_2 \right) /5 $, so the probability of $\psi_1$ is $0.6^2=0.36$ and the probability of $\psi_2$ is $0.8^2=0.64$. 

 {\noindent\bf Question 5c.} The chances of measuring $\phi_1$ and then $\psi_1$ again are $0.36\cdot 0.36=0.1296$ while the probability of measuring $\phi_2$ and then $\psi_2$ are $0.64\cdot 0.64=0.4096$, so the total probability is $0.5392$. 


\end{document}
