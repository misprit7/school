\documentclass[letterpaper, reqno,11pt]{article}
\usepackage[margin=1.0in]{geometry}
\usepackage{color,latexsym,amsmath,amssymb,graphicx, float}
\usepackage{hyperref}

\hypersetup{
colorlinks=true,
linkcolor=magenta,
filecolor=magenta,
urlcolor=cyan,
}

\graphicspath{ {images/} }

\begin{document}
\pagenumbering{arabic}
\title{MATH 305}
\date{07/03/22}
\author{Xander Naumenko}
\maketitle

\noindent
1. Use Fundamental Theorem of Calculus to evaluate

(a) $\int_{C} e^z dz$, $C:$ arc $ e^{it}, -\frac{\pi}{2} \leq t \leq  \pi$

Let $F(z)=e^{z}$. Then $\frac{d}{dz}F=e^{z}$, so by FTC: 
\[
\int_C e^{z}dz=e^{e^{i\pi}}-e^{e^{-\frac{\pi}{2}i}}=\frac{1}{e}-e^{-i}
.\]

(b)   $\int_{C} \frac{1}{z} dz$, $C:$ part of the ellipse $ \frac{x^2}{4}+ y^2=1, x \geq 0$

Let $F=Log z$. Then by FTC taking the contour to be slightly less than all the way around the circle: 
 \[
\int_C \frac{1}{z}dz=\frac{1}{2}\cdot2\pi i-0=\pi i
.\]

(c) $ \int_{C} \frac{1}{ z^2} dz$, $C:$ part of the ellipse $ \frac{x^2}{4}+ y^2=1, y \geq 0$.

Let $F=-\frac{1}{2z}$. Then by FTC: 
\[
\int_C \frac{1}{z^2}dz=\frac{1}{2}+\frac{1}{2}=1
.\]



\medskip

\noindent
2. (15pts)  Use the inequality $ |\int_{\Gamma} f(z) dz| \leq \max_{z \in \Gamma} |f(z)| \times \mbox{length of} ( \Gamma)$ to prove

(a) $ | \int_{ C} \frac{dz}{ z^2-i} | \leq \frac{3\pi}{4}$, $C$: circle $ |z|=3$ traversed once

The maximum of $\frac{1}{z^2-i}$ over the circle is when $z=\pm 3e^{i\frac{\pi}{4}}$, where $|f(z)|=\frac{1}{8}$ (this is obvious geometrically). Then
\[
\left|\int_C \frac{1}{z^2-i}dz\right|=\frac{1}{8}\cdot 6\pi=\frac{3\pi}{4}
.\]

(b) $ |\int_{C} Log (z) dz | \leq \frac{\pi^2}{{\bf 2}}$, $C:$ arc $ e^{it}, -\frac{\pi}{2} \leq t \leq \frac{\pi}{2}$

The maximum is attained when $z=e^{\pm i\frac{\pi}{2}}$ where $|Log(z)|=\frac{\pi}{2}$, so we have: 
\[
\left|\int_C Log(z)dz\right|\leq \frac{\pi}{2}\cdot \pi=\frac{\pi^2}{2}
.\]

(c) $ |\int_{C} \frac{e^{3z}}{e^z+1} dz|\leq \frac{2\pi e^{3R}}{e^R-1}$, $C$ is the vertical line segment from $z=R (>0)$ to $ z= R+ 2\pi i $.

For the numerator the imaginary component doesn't matter for the magnitude, and for the denominator the magnitude is minimized when $Im(z)=\pi$. Thus by FTC:
\[
\left|\int_{C} \frac{e^{3z}}{e^{z}+1}dz\right|\leq 2\pi\cdot \frac{e^{3R}}{e^{R}-1}
.\]



\medskip

\noindent
3. (15pts) Show that

(a) $ \int_{C_\epsilon } \frac{ Log (z) }{1+z^2} dz  \to 0 $ as $\epsilon \to 0$, where $ C_\epsilon  $ is the contour $ \epsilon e^{i t}, -\pi + \epsilon  \leq  t \leq  \pi -\epsilon $

Bounding the limit: 
\[
\left|\frac{Log(z)}{1+z^2}\right|\leq \frac{|\ln\epsilon+i\pi|}{|z|^2-1}
.\]
Then we can bound the integral as: 
\[
\left|\int_{C_\epsilon}\frac{Log(z)}{1+z^2}dz\right|\leq \frac{|\ln\epsilon+i\pi|}{||z|^2-1|}2(\pi-\epsilon)\epsilon
.\]
\[
\lim_{\epsilon\to 0}\frac{\sqrt{(\ln\epsilon)^2+\pi^2}}{1-\epsilon^2}2(\pi-\epsilon)\epsilon=2\pi\lim_{\epsilon\to 0}\epsilon \ln\epsilon=2\pi\lim_{\epsilon\to 0}\frac{\ln\epsilon}{1 /\epsilon}=2\pi\lim_{\epsilon\to 0}\epsilon=0
.\]
The last step was by L'Hopital's rule, and we're done. 

(b) $ \int_{C_R } \frac{ Log (z) }{1+z^2} dz  \to 0 $ as $R \to +\infty $, where $ C_R $ is the contour $ R e^{i t}, -\pi + \frac{1}{R}   \leq  t \leq  \pi -\frac{1}{R} $;

Using the exact same bounds for the integral as last time except with $\epsilon=\frac{1}{R}$, we get: 

\[
\left|\int_{C_R}\frac{Log(z)}{1+z^2}dz\right|\leq \frac{|-\ln R+i\pi|}{||z|^2-1|}2(\pi-\frac{1}{R})R
.\]
\[
\lim_{R\to \infty}\frac{|-\ln R+i\pi|}{|R^2-1|}2(\pi-\frac{1}{R})R=2\pi\lim_{R\to \infty}\frac{\ln R}{R}=2\pi\lim_{R\to \infty}\frac{1}{R}=0
.\]


\medskip

\noindent
4. Use Fundamental Theorem of Calculus to compute

(a) $ \int_{ \Gamma} z^{\frac{1}{2}} dz $ for the principal branch  of $ z^{\frac{1}{2}}$, where $ \Gamma $ is $ r= 2 \cos \frac{\theta}{2}, -\frac{\pi}{2} \leq \theta \leq \frac{\pi}{2}$

Let $F=\frac{2}{3}z z^{\frac{1}{2}}$. Then $F'(x)=\frac{2}{3}z^{\frac{1}{2}}+\frac{1}{3}z^{\frac{1}{2}}=z^{\frac{1}{2}}$. Then using FTC we get: 
\[
\int_{\Gamma}z^{\frac{1}{2}}dz=\frac{2}{3}z z^{\frac{1}{2}}\bigg|_{\sqrt{2}e^{-i\frac{\pi}{2}} }^{\sqrt{2}e^{i\frac{\pi}{2}} }=\frac{2}{3}\left( 2^{3 /4}e^{i\frac{3\pi}{4}}-2^{3 /4}e^{-i\frac{3\pi}{4}} \right)=\frac{2^{11 /4}}{3}i\sin(\frac{3\pi}{4})=\frac{2^{9/4}}{3}i
.\]

(b) $ \int_{\Gamma} (Log (z))^2 dz$, where $ \Gamma $ is $ r= 2 \cos \frac{\theta}{2}, -\frac{\pi}{2} \leq \theta \leq \frac{\pi}{2}$

Let $F(z)=zLog(z))^2$. Then as the hint suggests note that $( z (Log (z))^2)^{'} = (Log (z))^2 + 2 Log (z)$, so we have that

\[
    \int_{\Gamma} (Log(z))^2dz=z(Log(z))^2\bigg|_{\sqrt{2}e^{-i\frac{\pi}{2}} }^{\sqrt{2}e^{i\frac{\pi}{2}} }-2\int_\Gamma Log(z)dz
\]
\[
=\sqrt{2}i\left( \frac{1}{2}\ln 2+i\frac{\pi}{2} \right)^2 +\sqrt{2}i\left( \frac{1}{2}\ln 2-i\frac{\pi}{2} \right)^2-2\left( zLog(z)-z \right)\bigg|_{-\sqrt{2} i}^{\sqrt{2} i}
.\]
\[
=\sqrt{2}i\left( \frac{1}{2}\ln 2+i\frac{\pi}{2} \right)^2 +\sqrt{2}i\left( \frac{1}{2}\ln 2-i\frac{\pi}{2} \right)^2-2\left( \sqrt{2} i\left( \frac{1}{2}\ln 2+i \frac{\pi}{2}-1 \right) +\sqrt{2} i\left( \frac{1}{2}\ln 2-i \frac{\pi}{2}-1 \right) \right) 
.\]




\medskip

\noindent
5. Let $C$ be the contour of ellipse $ \frac{x^2}{4}+ y^2=1$ traversed once. Compute

(a) $ \int_C \frac{ 1 }{ (z-1)^2 } dz$

Applying Cauchy's integral formula with $f=1$, this gives: 
\[
\int_C \frac{1}{(z-1)^2}dz=2\pi if'(1)=0
.\]

(b) $ \int_C \frac{ e^z }{ z (z-1) } dz$

\[
\int_c \frac{e^{z}}{z(z-1)}dz=2\pi i e^{z} \frac{1}{z}\bigg|_{z=1} +2\pi i e^{z}\frac{1}{z-1}\bigg|_{z=0}=2\pi ie-2\pi i
.\]

(c) $ \int_C \frac{ 1 }{ z (z^2-1) } dz$

\[
\int_c \frac{1}{z(z^2-1)}dz=2\pi i  \frac{1}{z(z-1)}\bigg|_{z=-1}+2\pi i  \frac{1}{z(z+1)}\bigg|_{z=1} +2\pi i \frac{1}{(z+1)(z-1)}\bigg|_{z=0} 
\]
\[
=\pi i+\pi i -2\pi i=0
.\]

(d) $ \int_{C} \frac{1}{ 2z^2+1} dz $

\[
\int_{C} \frac{1}{ 2z^2+1} dz=\frac{2\pi i}{\left(\sqrt{2} z+i  \right) }\bigg|_{i/\sqrt{2} }+\frac{2\pi i}{\left(\sqrt{2} z-i  \right) }\bigg|_{-i /\sqrt{2} }=\frac{2\pi i}{2i}-\frac{2 \pi i}{2i}=0
.\]


\medskip

\noindent
6. Determine the domain of the analyticity of the following function and explain why
$$ \int_{|z|=2} f(z) dz=0 $$

(a) $f(z)=\frac{\cos z}{ z^2+6z+10}$

The domain of analyticity is $\mathbb{C}\setminus \{ z=3\pm i \} $. The integral is zero because the function is analytic in the domain and so the Cauchy integral formula tells us the integral is zero. 

(b) $ f(z) = \mbox{Log} (2z+5)$

The domain of analyticity is when $Re(2z+5)\geq 0, Im(2z+5)\neq 0$, i.e. $\mathbb{C}\setminus \{z\in\mathbb{C} : Re(z)\leq -\frac{5}{2}, Im(z)=0\} $. The integral is zero because again, the function is function is analytic within the circle and the Cauchy integral formula says it must be zero. 

(c) $ f(z)= \sin^{-1} (\frac{z}{3})$

Expanding: 
\[
\sin^{-1}(z)=-iLog\left( \frac{iz}{3}+\sqrt{ (1-\frac{z}{3}^2)} \right) 
.\]
This gives us that the domain of analyticity is $\mathbb{C}\setminus(-\infty, -3]\cup [3, \infty) $. The integral is zero since the function is analytic in the circle of radius two and Cauchy's integral formula. 

(d) $ f(z)= \tan (\frac{z}{2}) $

This function is analytic when $\cos(z)\neq 0$, i.e. $ \mathbb{C}\setminus\{z\in\mathbb Z\mid z=2\pi n+\pi, n\in\mathbb{N}, Im(z)=0\} $. The integral is zero since the function is analytic in the domain. 

\medskip





\noindent
7. Evaluate the contour integral $\int_{C} \frac{z}{ (z^2+1)(z-1)} dz$ along the following contours

(a) $C: | z-i|=1$, counter-clockwise

\[
\int_{C} \frac{z}{ (z^2+1)(z-1)} dz=2\pi i\frac{z}{(z+i)(z-1)}\bigg|_{z=i}=\frac{\pi i}{(i-1)}
.\]

(b) $C= C_1 \cup C_2$, $C_1: |z-i|=1$,counter-clockwise; $C_2= |z+i|=1$,clockwise

\[
\int_{C} \frac{z}{ (z^2+1)(z-1)} dz=2\pi i\frac{z}{(z+i)(z-1)}\bigg|_{z=i}-2\pi i\frac{z}{(z-i)(z-1)}\bigg|_{z=-i}=\frac{\pi i}{(i-1)}-\frac{\pi i}{-i-1}
.\]

(c) $C= C_1 \cup C_2 $, $C_1: |z-1|=1$, counter-clockwise; $C_2:  |z+1|=1$, clockwise.

\[
\int_{C} \frac{z}{ (z^2+1)(z-1)} dz=-2\pi i\frac{z}{(z-i)(z+i)}\bigg|_{z=1}=-\frac{2 \pi i}{(i+1)(i-1)}=\pi i
.\]



\medskip


\noindent
8. Evaluate the contour integral $ \int_{C} \frac{2z^2-z+1}{ (z-1)(z+1)^2} dz $ along the contour $C= C_1 \cup C_2 $, where $C_1: |z-1|=1$, counter-clockwise; $C_2:  |z+1|=1$, clockwise.


Hint: you can do partial fractions first.
\medskip

First note that
\[
\frac{2z^2-z+1}{(z-1)(z+1)^2}=\frac{A}{z-1}+\frac{Bz+C}{(z+1)^2}=\frac{1 /2}{z-1}+\frac{3z /2-1 /2}{(z+1)^2}
.\]
Applying the Cauchy integral formula this gives us: 
\[
\int_{C}\frac{2z^2-z+1}{(z-1)(z+1)^2}dz=\int_C\frac{1 /2}{z-1}+\frac{3z /2-1 /2}{(z+1)^2}dz
.\]
\[
=\frac{1}{2} 2\pi i-\frac{3}{2}2\pi i=-2\pi i
.\]


\noindent
9. Evaluate

(a) $\int_{|z|=2} \frac{1}{ z^2 +2z +2} dz$

The roots of the polynomial are $z=-1\pm i$, so the integral is: 
\[
\int_{|z|=2} \frac{1}{ z^2 +2z +2} dz=\frac{2\pi i}{z+1+i}\bigg|_{z=-1+i}+\frac{2\pi i}{z+1-i}\bigg|_{z=-1-i}
.\]
\[
=\pi-\pi=0
.\]

(b) $ \int_{|z|=2} \frac{ 1}{ z^2 -2z-3} dz $

The roots are $z=3, z=-1$, so: 
 \[
\int_{|z|=2} \frac{ 1}{ z^2 -2z-3} dz=\frac{2\pi i}{z-3}\bigg|_{z=-1}=-\pi \frac{i}{2}
.\]

\end{document}
