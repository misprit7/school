\documentclass{article}

\usepackage{graphicx}
\usepackage{setspace}

\topmargin -35 mm
\addtolength{\topmargin}{18 true mm}
\oddsidemargin-45 mm
\addtolength{\oddsidemargin}{40 true mm}
\evensidemargin-5 mm
\textwidth 180 true mm
\textheight 240 true mm


\newcommand{\e}[1]{\vec{e}_{#1}} 
\newcommand{\x}[1]{\dot{#1}} 
\newcommand{\xx}[1]{\ddot{#1}} 
\newcommand{\p}[1]{\partial #1}
\newcommand{\EqLabel}[1]{\label{#1}}

\begin{document}

\begin{center}{\Large {\bf Problem set 2 (due Wed. Feb 9 by 4pm)}}
\end{center}

\large


1. Derive the Euler-Lagrange equations from Hamilton's principle, for
a system with $s>1$ degrees  of freedom. {\bf Hint:} in class we did the derivation for
$s=1$.

\vspace{3mm}

\noindent\begin{minipage}[b]{80mm}
\centering
\includegraphics[width=80mm]{fig2.eps}
\small
\\Figure for problem 2.  
\end{minipage}
\hfill
\begin{minipage}[b]{90mm}
2. Write the Lagrangian for the system  shown in Fig. 2, using $x$ and $l$ as
   generalized coordinates. Find the 
   Euler-Lagrange equations. Use them to find the accelerations of the
   two bodies. Do you get the same answers as last time?

   {\bf For fun}: Assume that both objects start at rest, at $t=0$. At what time $t$ has the mass $m$ slid a distance $L$ down the incline (we assume it started high enough that it is still on the incline)? By how much has the mass $M$ moved during this interval?
\end{minipage}

\vspace{5mm}

\noindent
\begin{minipage}[b]{60mm}
\includegraphics[width=60mm]{fig11.eps}
\\{\small Figure for problem 3.}
\end{minipage}
\hfill
\begin{minipage}[b]{120mm}
  {\bf 3. From last year's 1st midterm:} Consider the system shown in Fig. $\leftarrow$ . Mass $m_1$ moves only along the vertical; it is tied to the ceiling with a  spring that has an unstretched length $l_0$ and a spring constant $k$. On the other side, a massless rope is tying $m_1$ to $m_3$, going around the two pulleys as shown (the pulleys have no mass). Mass $m_3$ can slide without friction along the fixed incline of angle $\alpha$. All other quantities that you might need (perhaps the radii of the pulleys, or the length of the rope, etc.) are known and you can give them whatever names you wish.

  (i) Find the Lagrangian.

  (ii) Find the corresponding Euler-Lagrange equation(s).

\end{minipage}

\vspace{5mm}

\noindent\begin{minipage}[c]{120mm}
{\bf 4. From last year's 1st midterm:}.
Consider the system shown in that $\rightarrow$ figure. Mass $m_1$ can slide without friction along the horizontal line, and it is tied to mass $m_2$ through a planar pendulum of length $l$. The whole system is in an elevator that is moving upwards with a known constant speed $V\ne 0 $.

(i) Find the Lagrangian. Important: discard from ${\cal L}$ all the terms that do not contribute to the Euler-Lagrange equation(s). 

(ii) Find all conserved quantities.

(iii) Explain how one can solve the problem, once we're given initial conditions. It is fine if some of the answers are in the form of integrals (do not attempt to do these integrals). 


\end{minipage} 
\hfill
\begin{minipage}[c]{50mm}
\centering
\includegraphics[width=50mm]{fig22.eps}
\\{\small Figure for problem 4.}
\end{minipage}


\vspace{5mm} 

\noindent
\begin{minipage}[b]{50mm}
\includegraphics[width=50mm]{fig33.eps}
\\{\small Figure 3 (problem 3).}
\end{minipage}
\hfill
\begin{minipage}[b]{130mm}
  {\bf 5. From last year's 1st midterm:}. Consider the system shown in Fig. 3, where mass $m_1$ is constrained to move vertically, while $m_3$ is constrained to move horizontally. They are connected by a massless rod of length $L$, on which there is a mass $m_2$ that can slide without friction.

  (i) Find the Lagrangian.

  (ii) Indicate two ways to verify that the answer at (i) is reasonable. 
  
  (iii) Find all the conserved quantities.

\end{minipage}

\vfill

{\bf The other side has more problems (for fun, not graded)}

\newpage

{\bf Additional problem: (not graded)}

\vspace{10mm}

1.  Let ${\cal L}(q,\x q, t)$ be the Lagrangian of a system with one
   degree of freedom, and let ${\cal L}'(q,\x q, t) = {\cal L}(q,\x q,
   t) + {d \over dt} f(q)$, where $f(q)$ is some function. Use the
   Euler-Lagrange equations to show that the dynamics of the two systems is
   identical, i.e.
$$ 
   {d \over dt} \left(\frac{\p {\cal L}}{\p \x q} \right) - \frac{\p
   {\cal L}}{\p q} = 0 \longleftrightarrow  {d \over dt}
   \left(\frac{\p {\cal L}'}{\p \x q} \right) - \frac{\p  {\cal
   L}'}{\p q} = 0 
$$ 

   Now prove the invariance for a general function $f(q,t)$. If you want, redo for a system with more degrees of freedom.


\vspace{10mm}
   
2. Show that for a particle constrained to move along
the $x$ axis from $x(t_1)=x_1$ to $x(t_2)=t_2$, and which is otherwise free,
the action really has
a minimum for the true path $x(t)$. Note: since there are no
interactions, the  Lagrangian is ${\cal L} = {m\x x^2\over 2}$.



\vspace{10mm}

3. Relativistic mechanics: consider the Lagrangian ${\cal L}(x,\x x) =
   - m_0 c^2 \sqrt{1-\x x^2/c^2} - U(x)$. Show that the equation of
   motion of this system (the Euler-Lagrange equation) is:
$$
{d  p \over dt} = F, \mbox{      where    }
p = m \x x = { m_0 \over \sqrt{1-\x x^2/c^2}} \x x \mbox{ and } F = -{
  dU\over dx}
$$
You should be able to figure out how to generalize this to three-dimensional motion, so basically now you know what special relativity looks like in Lagrangian formalism. You can also check that in the limit where the speed of the particle is very much smaller than the speed of light $c$, this relativistic Lagrangian agrees with the non-relativistic one we're using. 



\vspace{10mm}

4. 
 The Lagrangian of a non-relativistic particle of mass $m$ and charge $q$ moving in a constant magnetic field
   $\vec{B}=B\e z$ is ${\cal L} =  {m \over 2} (\x x^2 + \x y^2 + \x
   z^2) + q \x y Bx$. Derive the Euler-Lagrange equations and show that
   they agree with the Newtonian formulation. 

   More general formulation: The general Lagrangian of a non-relativistic particle of
charge $q$ moving in a magnetic field $\vec{B}(\vec{r})$ is ${\cal L}
= m\vec{v}^2/2 + q \vec{v}\cdot\vec{A}(\vec{r})$, where
$\vec{A}(\vec{r})$ is the vector potential: $\vec{B} = \nabla \times
\vec{A}$. Derive  the Euler-Lagrange equations in this case. 

\end{document}
