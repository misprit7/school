\documentclass[letterpaper, reqno,11pt]{article}
\usepackage[margin=1.0in]{geometry}
\usepackage{color,latexsym,amsmath,amssymb,graphicx,float,listings,tikz}
\usepackage{hyperref}

\hypersetup{
colorlinks=true,
linkcolor=magenta,
filecolor=magenta,
urlcolor=cyan,
}

\graphicspath{ {images/} }

\begin{document}
\pagenumbering{arabic}
\title{Math 437 Homework 1}
\date{28/09/23}
\author{Xander Naumenko}
\maketitle

{\medskip\noindent\bf Question 1.} Let $m\in \mathbb{N}$, and let $s_i=\frac{i}{\sqrt{i+1}}$. $s_i$ is clearly unbounded since $\frac{i}{\sqrt{i+1}}= \frac{i+1}{\sqrt{i+1}}-\frac{1}{\sqrt{i+1}}\geq\sqrt{i+1}-1$ is unbounded. Also note that $\frac{s_{i+1}}{s_i}=\frac{(i+1)\sqrt{i+1}}{i\sqrt{i+2}}\geq \frac{i+1}{i}>1$, so $s_i$ is strictly increasing. Thus there exists a smallest number $n\in\mathbb{N}$ with $m\leq s_n$.

I claim that $m=\left[\frac{n}{\sqrt{n+1}}\right]$. To see why suppose by contradiction not, since $m\leq \frac{n}{\sqrt{n+1}}$ this would mean $\frac{n}{\sqrt{n+1}}\geq m+1$. But then $s_{n-1}=\frac{n-1}{\sqrt{n}}\geq \frac{n-\sqrt{n+1}}{\sqrt{n+1}}\geq m$, which contradicts the assumption that $n$ was the smallest possible. Thus it must be that $m=\left[\frac{n}{\sqrt{n+1}}\right]$.

\medskip

{\medskip\noindent\bf Question 2.} Let $A$ be the union of finitely many arithmetic progressions with coefficients $(a_1,b_1), \ldots (a_N,b_N)$ respectively. Let $m=\text{lcm}\left( a_1,a_2,\ldots,a_N \right) $ and $B=\max(b_1, b_2, \ldots b_N) $. Then for each $k\in \{B,B+1,\ldots B+m-1\}$, if $k\in S$ then define an arithmetic progression with $a_k'=m,b_k'=k$. Let $S'$ be the union of all arithmetic progressions generated this way, along with the finite set $\{n\in S:n<B\}$.

I claim $S=S'$. Let $c\in S$, if $c< B$ then by construction $c\in S'$. Otherwise, by the division algorithm let $c-B=mn+r, n\geq0, 0\leq r<m$. $B+r\in S$, since if it wasn't then the progression $(a_i,b_i)$ with $B+r = a_i n+b_i$ would also include $c$ since $c=(B+r)+mn$ and $a_i|m$. Thus $S\subseteq S'$.

For the other direction, let $d\in S'$. Again if $d< B$ then by construction $d\in S$ automatically. Otherwise, let $a_i,b_i$ be two coefficients used to generate $S$ originally, and since $d\in S'$ it can be written as $d=mn'+B+r$ for some $n'\geq 0,r\in \{0,1,\ldots,m-1\}$. Note that there exists no $n\geq 0$ with $a_i n+b_i=B+r$ by the definition of $S'$. Now by contradiction suppose there exists a solution $n$ to the equation $d=mn+B+r=a_i n'+b_i$. Then $a_i(n'-\frac{m}{a_i}n)+b_i=B+r$, but we just stated this was impossible by the construction of $S'$, so no such solutions to $d=a_in'+b_i$. Since this holds for all choices of $a_i,b_i$ we get that $d\in S$, so $S'\subseteq S$. Since both sets contain one another, $S=S'$ as required.

\medskip

{\medskip\noindent\bf Question 3.} Note that finding solutions to the given problem is finding solutions for
\[
3m+1=nk
\]
\[
n^2+3=ml
\]
for odd positive integers $m,n$ and $k,l\in\mathbb{N}$. I will split solutions into cases: $k=1$,$k=2$, $k=3$ and $k>3$. I will handle them in backwards order.

\medskip

{\noindent\bf Case $k>3$:} Assume $k>3$, so $3m+1=kn\leq 4n\implies m>n$ or $m=n=1$. Also, taking the first equation and plugging it into the second gets
\[
3n^2+9=nkl-l\implies n|l+9
.\]
Thus either $n<l$, $l\leq n<l+9$ or $n=l+9$.

{\bf Subcase $k>3,n<l$:} In this case we have $m\geq n+1,l\geq n+1\implies n^2+3=ml\geq (n+1)^2=n^2+2n+1\implies n=1\implies m|4\implies m=1,2$ or $4$, but since $m$ is odd the only possibility is $m=n=1$.

{\bf Subcase $k>3,l\leq n<l+9$:} Note that in this case $n\leq 9$, since otherwise it's clearly impossible for $n|l+9$. To find all the possibilities, all the possible values of $n=0,1,\ldots,8$ can be manually checked for possible values of $l$ fulfilling the conditions. Doing so gives the possible values of pairs: $\left( n,l \right)\in \{(2,1),(3,3),(4,3),(5,1),(6,3),(7,5),(8,7),(9,9)\}$. Of course once $n$ and $l$ are specified there are two equations with two unknowns, so $m,k$ are uniquely specified. Here is a Python script to write check for these cases, remembering also the parity requirement:
\begin{lstlisting}
for (n,l) in [(2,1),(3,3),(4,3),(5,1),(6,3),(7,5),(8,7),(9,9)]:
    m = (n**2+3)//l
    if (n**2+3)%l == 0 and (3*m+1)%n == 0 and m%2 == 1 and n%2 == 1:
        print(n,l)
\end{lstlisting}
Since it produces no output, none of these are possibilities.

{\bf Subcase $k>3,n=l+9$} Given $n=l+9$, we have $n\geq 10$ and we get that:
\[
ml=m(n-9)=n^2+3\implies m=\frac{n^2+3}{m-9}
\]
\[
\implies k=3 \frac{n^2+3}{n(n-9)}+\frac{1}{n}=3\left( \frac{n+9}{n}+\frac{84}{n(n-9)}+\frac{1}{n} \right) =3+\frac{28}{n}+ \frac{252}{n(n-9)}\leq 3+3+26=32
.\]
Since this gives a bound on $k$ and additionally we have that $n=l+9$, we can iteratively go through $k=1,2,\ldots,32$, and for each one there is once more a system of two equations with two unknowns, which is easily checkable. These equations are:
\[
3m+1=nk
\]
\[
n^2+3=m(n-9)
\]
\[
\implies m=\frac{3k^2+1}{k-3},n=\frac{9k+1}{k-3}
.\]
This is a Python script to iterate over $k$ to find all solutions of this form:
\begin{lstlisting}
for k in range(4,33):
    if (3*k**2+1)%(k-3) == 0 and (9*k+1)%(k-3) == 0:
        m,n = ((3*k**2+1)//(k-3), (9*k+1)//(k-3))
        if m%2 == 1 and n%2 == 1:
            print(m,n)
\end{lstlisting}
The output of this program shows that the only odd $m,n$ of this form are $m=49,n=37$ and $m=43,n=13$.

\medskip

{\noindent\bf Case $k=3$:} In this case $3m+1=3n$, but the left side is divisible by $3$ while the right isn't, so this case is impossible.

\medskip

{\noindent\bf Case $k=2$:} As already stated previously $3n^2+9=nkl-l$, and in this case because $k=2$ we can rearrange to get
\[
l=\frac{3n^2+9}{2n-1}\leq \frac{12n^2}{n}=12n
.\]
Also, since $n|l+9$ we can write $l+9=na$ for some $a\in \mathbb{N}$. Using the above bound gives us that $a=\frac{l+9}{n}\leq \frac{12n+9}{n}\leq 21$ (I'm being very sloppy with the bounds here and it's certainly possible to be clever and get better, but since this will be computer checked any ways it's not too important to be super precise). As already done previously, each possibly value for $a=1,\ldots,21$ can be tried and the resulting system solved. Rearranging the original equations with the fact that $l=an-9$ and $k=2$:
\[
m=\frac{13}{2a-3},n=\frac{a+18}{2a-3}
.\]
Here is a Python script that checks for solutions:
\begin{lstlisting}
for a in range(1,21):
    if 13%(2*a-3) == 0 and (a+18)%(2*a-3) == 0:
        m,n = (13//(2*a-3), (a+18)//(2*a-3))
        if m%2 == 1 and n%2 == 1 and m > 0 and n > 0:
            print(m,n)
\end{lstlisting}
Since it produces no output, this case is impossible.

\medskip

{\noindent\bf Case $k=1$:} In this case we have $3m+1=n$. However for $m,n$ odd, the left side is even while the right side is odd and thus equality can't hold, so this case is impossible.

\medskip

Since all possible values of $k$ have been covered, the only possible odd values of $m$ and $n$ that fulfill the given properties are $m=n=1$, $m=32,n=13$, and $m=49,n=37$.


\end{document}
