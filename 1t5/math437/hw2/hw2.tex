\documentclass[letterpaper, reqno,11pt]{article}
\usepackage[margin=1.0in]{geometry}
\usepackage{color,latexsym,amsmath,amssymb,graphicx,float,listings,tikz}
\usepackage{hyperref}

\hypersetup{
colorlinks=true,
linkcolor=magenta,
filecolor=magenta,
urlcolor=cyan,
}

\graphicspath{ {images/} }

\begin{document}
\pagenumbering{arabic}
\title{Math 437 Homework 2}
\date{16/10/23}
\author{Xander Naumenko}
\maketitle

{\medskip\noindent\bf Question 1.} Since $1|n\forall n$, we have that $3|n+2\implies n\equiv 1\mod 3$. Clearly any prime of the form $3k+1$ works and no other prime does, since for such numbers $1$ is the only factor. I claim that primes of that form are the only solution for $n$. 

Proof by contradiction, assume that $n=pa$ where $p<n$ is the smallest prime divisor of $n$, with $p+2|n+2$ and $a+2|n+2$. If $2|n$ then we have $\frac{n}{2}+2|n+2$, which is impossible since $\frac{n+2}{2}<\frac{n}{2}+2<n+2$ so $p\neq 2$. Next consider the set of congruence relations:
\[
    \begin{cases}
        n'\equiv 0&\mod p\\
        n'\equiv -2&\mod p+2
    \end{cases}
.\]
$p$ is odd so $\gcd(p,p+2)=1$, so by the chinese remainder theorem the solution $n'$ is unique up to multiples of $p(p+2)$. $n'=p$ fulfills both criteria, so we can express $n=p+kp(p+2),k\in \mathbb{Z}$.

Now consider $a=\frac{n}{p}=1+k(p+2)$. By hypothesis $a+2|n+2\implies (3+k(p+2))|(p+kp(p+2)+2)\implies (1+k(p+2))|2$. Clearly this is impossible for $p>1$ which it is, so this is a contradiction suggesting $n$ can't in fact have more factors than $1$ and itself. Since we've shown that primes of the form $3k+1$ work and any composite numbers don't, all primes of that form are the only numbers that fulfill the requirements. $\square$

{\medskip\noindent\bf Question 2.} Consider the equation mod 3:
\[
2^{m}\equiv 1\mod 3\implies m=2k, k\in \mathbb{N}
.\]
Now consider the same equation mod 4:
\[
4^{k}-3^{n}\equiv -3^{n}\equiv 3\mod 4\implies n=2l, l\in \mathbb{N}
.\]
But then the equation reduces to $4^{k}-9^{l}=(2^{k}+3^{l})(2^{k}-3^{l})=7$. Since 7 is prime this means that $2^{k}+3^{l}=7$, $2^{k}-3^{l}=1$. Since $2^{k}+3^{l}$ is clearly increasing in $k,l$ it's trivial to check the possibilities $k=1,2,l=1$ and see that the only solutions correspond to $m=4,n=2$. $\square$

{\medskip\noindent\bf Question 3.} By theorem 13.4, we know that for a number $n$, it is expressible as $a^2+b^2$ if and only if the exponent its prime factors in the form $4l+3$ is even. There are infinitely prime numbers of the form $4l+3$, as if there were finitely many of them $4k_1+3,4k_2+3\ldots, 4k_m+3$, then we would have that $4(4k_1+3)\cdots (4k_m+3)+3$ isn't divisible by any of them but is of the form $4l+3$. It's prime factors can't be just of the form $4l+1$ as $(4l_1+1)(4l_2+1)=4(4l_1l_2+l_1+l_2)+1$, so at least on of its prime factors wasn't included on our supposedly complete list, implying there are infinitely many.

Using the fact that there are infinitely many take $q_0,\ldots, q_{k-1}$ to be arbitrary distinct primes of the form $4l+3$. Using the chinese remainder theorem, there exists a unique solution to the following system of equations:
\[
\begin{cases}
    
    x\equiv 0&\mod q_0\\
    x\equiv -1&\mod q_1\\
\vdots\\
    x\equiv -k+1&\mod q_{k-1}\\
\end{cases}
\]
up to mod $q_1\cdots q_{k-1}$. Let $m_i=1$ if $\exp_{q_i}(x+i)\equiv 0\mod 2$ and $m_i=\exp_{q_i}(x+i)+1$ otherwise. I claim that the following sequence of $k$ integers satisfies the required properties, where $n$ ranges from 0 to $k-1$:
\[
    x_n=x+n+\prod_{i=0}^{k-1}q_i^{m_i}
.\]
Note that the product term does not conflict with the congruence relations found above, since it is a multiple of $q_1\cdots q_{k-1}$. Consider any individual sequence element $x_n$. If $\exp_{q_{n}}(x+n)\equiv 0\mod 2$, then we can write $x+n=q_n^2l$ (it can't be that $\exp_{q_n}(x_n)=0$ since $x$ was the solution to $x\equiv -n\mod q_n$) and $x_n=q_n(q_nl+q_0^{m_0}\cdots q_{k-1}^{m_{k-1}})$. Importantly $q_n$ does not divide the second part of the addition but does the first, so $\exp_{q_n}(x_n)=1$.

If instead $\exp_{q_{n}}(x+n)\equiv 1\mod 2$, then we can write $x+n=q_n^{m_n}l$ for $q_n\nmid l$, and $x_n=q_n^{m_n}(l+q_nq_0^{m_0}\cdots q_{k-1}^{m_{k-1}})$. In reverse from the previous case here the first term is not divisible by $l$ and the second is, so $\exp_{q_n}(x_n)\equiv m_n\equiv 1\mod 2$. In either case we have that $\exp_{q_n}(x_n)\equiv 1\mod 2$, so by theorem 13.4 none of the $x_n$ are expressible as $a^2+b^2$. $\square$

{\medskip\noindent\bf Question 4a.} I claim that the limit is equal to 0. Writing $n!=\prod_{i=1}^{r}p_i^{\alpha_i}$ with $p_1<p_2<\ldots<p_r$, using identities proven in class we have that
\[
d(n!)\phi(n!)=\left(\prod_{i=1}^{r}(\alpha_i+1)\right)n!\left(\prod_{i=1}^{r}\left(1-\frac{1}{p_i}\right)\right)=n!\left( \prod_{i=1}^{r}(\alpha_i+1)\left( 1-\frac{1}{p_i} \right)  \right) 
.\]
Since $n!|(n+1)!$, each individual term in the product above only increases as $n$ increases. Also since $\alpha_i\geq 1$ and $1-\frac{1}{p_i}\geq \frac{1}{2}$, each individual term in the product is greater or equal to 1. Thus:
\[
    \frac{n!}{d(n!)\phi(n!)}\leq \frac{n!}{n! \frac{\exp_2(n!)}{2}}=\frac{2}{\exp_2(n!)} \to 0
.\]

{\medskip\noindent\bf Question 4b.} This limit is also 0. Applying the ratio test to $x_n= \frac{n!}{2^{d(n!)}}$:
\[
\frac{(n+1)!}{2^{d((n+1)!)}} \frac{2^{d(n!)}}{n!}= \frac{n+1}{2^{d((n+1)!)-d(n!)}}
.\]
Let $n!=\prod_{i=1}^{r}p_i^{\alpha_i}$, where $p_1<p_2<\ldots<p_r$. Then as we showed in class we have $d(n!)=\prod_{i=1}^{r}(\alpha_i+1)$. Since $n!|(n+1)!$, if $n+1$ isn't a prime we can represent $d((n+1)!)=\prod_{i=1}^{r}p_i^{\alpha'_i}$, with $\alpha_i'\geq\alpha_i\forall i$ and strict inequality holding at least once. If $n+1$ is prime, then we have $d((n+1)!)=(n+1)d(n!)$. For $n\geq 2$, $n!$ is even so $p_1=2$, and since $2$ is the smallest prime, $\alpha_1\geq \alpha_i\forall i\in \mathbb{N}$. Therefore a lower bound for $d((n+1)!)$ regardless of whether $n+1$ is prime or not is $(\alpha_1+2)\prod_{i=2}^{r}(\alpha_i+1)$. Applying this:
\[
d((n+1)!)-d(n!)\geq \prod_{i=2}^{r}(\alpha_i+1)=\text{\# of odd divisors of $n!$}
.\]
Consider just divisors of $n!$ of the form $3^{k}5^{l}$ which is a subset of all odd divisors of $n!$. Based on the definition of factorial it's true that $\exp_3(n!)\geq \left\lfloor \frac{n}{3} \right\rfloor$ and $\exp_5(n!)\geq \left\lfloor \frac{n}{5} \right\rfloor$. Thus we have
\[
    \frac{x_{n+1}}{x_{n}}\leq \frac{n+1}{2^{\left\lfloor \frac{n}{3} \right\rfloor\left\lfloor \frac{n}{5} \right\rfloor}}\to 0
.\]
Thus by the ratio test the limit is zero.

\end{document}
