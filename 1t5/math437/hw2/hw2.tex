\documentclass[letterpaper, reqno,11pt]{article}
\usepackage[margin=1.0in]{geometry}
\usepackage{color,latexsym,amsmath,amssymb,graphicx,float,listings,tikz}
\usepackage{hyperref}

\hypersetup{
colorlinks=true,
linkcolor=magenta,
filecolor=magenta,
urlcolor=cyan,
}

\graphicspath{ {images/} }

\begin{document}
\pagenumbering{arabic}
\title{Math 437 Homework 2}
\date{16/10/23}
\author{Xander Naumenko}
\maketitle

{\medskip\noindent\bf Question 2.} Consider the equation mod 3:
\[
2^{m}\equiv 1\mod 3\implies m=2k, k\in \mathbb{N}
.\]
Now consider the same equation mod 2:
\[
4^{k}-3^{n}\equiv -3^{n}\equiv 1\mod 4\implies n=2l, l\in \mathbb{N}
.\]
But then the equation reduces to $4^{k}-9^{l}=(2^{k}+3^{l})(2^{k}-3^{l})=7$. Since 7 is prime this means that $2^{k}+3^{l}=7$, $2^{k}-3^{l}=1$. Since $2^{k}+3^{l}$ is clearly increasing in $k,l$ it's trivial to check the possibilities $k=1,2,l=1$ and see that the only solutions correspond to $m=4,n=2$. $\square$

{\medskip\noindent\bf Question 3.} By theorem 13.4, we know that for a number $n$, it is expressible as $a^2+b^2$ if and only if the exponent its prime factors in the form $4l+3$ is even. There are infinitely prime numbers of the form $4l+3$, as if there were finitely many of them $4k_1+3,4k_2+3\ldots, 4k_m+3$, then we would have that $4(4k_1+3)\cdots (4k_m+3)+3$ isn't divisible by any of them but is of the form $4l+3$. It's prime factors can't be just of the form $4l+1$ as $(4l_1+1)(4l_2+1)=4(4l_1l_2+l_1+l_2)+1$, so at least on of its prime factors wasn't included on our supposedly complete list, implying there are infinitely many.

Using the fact that there are infinitely many take $q_0,\ldots, q_{k-1}$ to be arbitrary distinct primes of the form $4l+3$. Using the chinese remainder theorem, there exists a unique solution to the following system of equations:
\[
\begin{cases}
    
x\equiv 0\mod q_0\\
x\equiv -1\mod q_1\\
\vdots\\
x\equiv -k+1\mod q_{k-1}\\
\end{cases}
\]
up to mod $q_1\cdots q_{k-1}$. Let $m_i=1$ if $\exp_{q_i}(x+i)\equiv 0\mod 2$ and $m_i=\exp_{q_i}(x+i)+1$ otherwise. I claim that the following sequence of $k$ integers satisfies the required properties, where $n$ ranges from 0 to $k-1$:
\[
    x_n=x+n+\prod_{i=0}^{k-1}q_i^{m_i}
.\]
Note that the product term does not conflict with the congruence relations found above, since it is a multiple of $q_1\cdots q_{k-1}$. Consider any individual sequence element $x_n$. If $\exp_{q_{n}}(x+n)\equiv 0\mod 2$, then we can write $x+n=q_n^2l$ (it can't be that $\exp_{q_n}(x_n)=0$ since $x$ was the solution to $x\equiv -n\mod q_n$) and $x_n=q_n(q_nl+q_0^{m_0}\cdots q_{k-1}^{m_{k-1}})$. Importantly $q_n$ does not divide the second part of the addition but does the first, so $\exp_{q_n}(x_n)=1$.

If instead $\exp_{q_{n}}(x+n)\equiv 1\mod 2$, then we can write $x+n=q_n^{m_n}l$ for $q_n\not |\ l$, and $x_n=q_n^{m_n}(l+q_nq_0^{m_0}\cdots q_{k-1}^{m_{k-1}})$. In reverse from the previous case here the first term is not divisible by $l$ and the second is, so $\exp_{q_n}(x_n)\equiv m_n\equiv 1\mod 2$. In either case we have that $\exp_{q_n}(x_n)\equiv 1\mod 2$, so by theorem 13.4 none of the $x_n$ are expressible as $a^2+b^2$. $\square$

\end{document}
