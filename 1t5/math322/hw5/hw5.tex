\documentclass[letterpaper, reqno,11pt]{article}
\usepackage[margin=1.0in]{geometry}
\usepackage{color,latexsym,amsmath,amssymb,graphicx,float,listings,tikz}
\usepackage{hyperref}

\hypersetup{
colorlinks=true,
linkcolor=magenta,
filecolor=magenta,
urlcolor=cyan,
}

\graphicspath{ {images/} }

\begin{document}
\pagenumbering{arabic}
\title{Math 322 Homework 5}
\date{10/10/23}
\author{Xander Naumenko}
\maketitle

{\medskip\noindent\bf Question 1.} Tracing through each element $1,2,\ldots,7$ through the cycles, we have that the given permutation is equivalent to
\[
    \begin{pmatrix} 1&2&3&4&5&6&7\\2&7&3&4&5&6&1 \end{pmatrix} 
.\]
Again manually tracing through the paths, we can factor this into $(127)(3)(4)(5)(6)$

{\medskip\noindent\bf Question 2.} Let $\alpha\in A_n$. Every permutation can be decomposed into a composition of transpositions, represent this as $\alpha=(ab)(cd)(ef)\cdots$. Note that for two adjacent transpositions in this decomposition $(ab),(cd)$we have $a\neq b,c\neq d$ and there can be at most one repeat of elements between the cycles (since if both were repeats then they would be the same transposition). Without loss of generality assume that $a\neq c,a\neq d$. Then note that $(ab)(cd)=(cad)(abc)$. This can be manually checked to be logically equivalent, and these are 3-cycles since $c,a,d$ and $a,b,c$ share no common elements between them for the reasons above. Since $\alpha\in A_n$, the number of 2-cycles in it's decomposition must be even by definition. Thus for each pair of adjacent transpositions we can convert them into an equivalent pair of 3-cycles as required.

{\medskip\noindent\bf Question 3.} Let $\alpha$ be the permutation given. Note that $(1n)(2,n-2)\cdots(\left\lfloor \frac{n}{2} \right\rfloor\left\lceil \frac{n}{2} \right\rceil)\alpha=1$, since each transposition on the right flips 2 elements back to their original places. The identity is even, and there are $\frac{n}{2}$ transpositions on the left if $n$ is even and $\frac{n-1}{2}$ if it's odd. Going through the all possibilities of $n\mod 4$, we find that
\[
\text{sg }\alpha=\begin{cases}
    1&\text{ if }n\equiv 0\mod 4\text{ or }n\equiv 1\mod 4\\
    -1&\text{ if }n\equiv 2\mod 4\text{ or }n\equiv 3\mod 4
\end{cases}
.\]

{\medskip\noindent\bf Question 4.} Let $m\in \{1,2,\ldots,n\}$. There are two cases: $\alpha^{-1}(m)\in \{i_1,i_2,\ldots,i_r\}$ or not. If so, then let $m=i_k$ we have
\[
\alpha(i_1i_2\cdots i_r)\alpha^{-1}m=\alpha(i_1i_2\cdots i_r)i_k=\alpha(i_{k+1})
.\]
This is clearly the same result as feeding $m=\alpha^{-1}(i_k)$ into the right side of the equation, so in this case the equation holds. Alternatively if $\alpha^{-1}(m)\notin \{i_1,i_2,\ldots,i_r\}$, $\alpha(i_1i_2\cdots i_r)\alpha^{-1}m=m=(\alpha(i_1)\alpha(i_2)\cdots \alpha(i_r))m$. In either case the equation holds, so it is true in general.

{\medskip\noindent\bf Question 5.} Every permutation in $S_n$ can be represented as a sequence of transpositions, so if it can be shown that any arbitrary transposition $(a,b)$ can be generated by just members of each set, the result follows immediately. Let $(a,b)$ be a transposition, $1\leq a,b\leq n$, without loss of generality assume that $a<b$. Then note that $(1b)(1a)(1b)=(a,b)$, so it can be represented by a finite product of elements from the first type of transpositions given. Also note that $(a,a+1)\cdots(b-2,b-1)(b-1,b)(b-2,b-1)\cdots(a,a+1)=(a,b)$, which represents $(ab)$ purely in terms of members of the second group. Thus both types of transpositions generate all possible transpositions, and thus every permutation.

\end{document}
