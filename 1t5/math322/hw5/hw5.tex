\documentclass[letterpaper, reqno,11pt]{article}
\usepackage[margin=1.0in]{geometry}
\usepackage{color,latexsym,amsmath,amssymb,graphicx,float,listings,tikz}
\usepackage{hyperref}

\hypersetup{
colorlinks=true,
linkcolor=magenta,
filecolor=magenta,
urlcolor=cyan,
}

\graphicspath{ {images/} }

\begin{document}
\pagenumbering{arabic}
\title{Math 322 Homework 5}
\date{10/10/23}
\author{Xander Naumenko}
\maketitle

{\medskip\noindent\bf Question 1.} Tracing through each element $1,2,\ldots,7$ through the cycles, we have that the given permutation is equivalent to
\[
    \begin{pmatrix} 1&2&3&4&5&6&7\\2&7&3&4&5&6&1 \end{pmatrix} 
.\]
Again manually tracing through the paths, we can factor this into $(127)(3)(4)(5)(6)$

{\medskip\noindent\bf Question 2.} Let $\alpha\in A_n$. Every permutation can be decomposed into disjoint cycles, write this as $\alpha=(i_1i_2\cdots i_r)\cdot (j_1j_2\cdots j_s)=(i_1i_r)\cdots(i_3i_1)(i_2i_1)$

{\medskip\noindent\bf Question 3.} Recall from class/the textbook that multiplying by a 

\end{document}
