\documentclass[letterpaper, reqno,11pt]{article}
\usepackage[margin=1.0in]{geometry}
\usepackage{color,latexsym,amsmath,amssymb,graphicx,float,listings,tikz}
\usepackage{hyperref}

\hypersetup{
colorlinks=true,
linkcolor=magenta,
filecolor=magenta,
urlcolor=cyan,
}

\graphicspath{ {images/} }

\begin{document}
\pagenumbering{arabic}
\title{Math 322 Homework 2}
\date{24/09/23}
\author{Xander Naumenko}
\maketitle

{\medskip\noindent\bf Question 1.} Simply using the definition of the maps and manually carrying through where each number gets mapped:

\[
    \alpha\beta=\begin{pmatrix} 1&2&3&4&5\\ 2&1&5&4&3 \end{pmatrix} 
.\]
\[
    \beta\alpha=\begin{pmatrix} 1&2&3&4&5\\ 3&4&1&2&5 \end{pmatrix} 
.\]
\[
    \alpha ^{-1}=\begin{pmatrix} 1&2&3&4&5\\ 3&1&2&5&4 \end{pmatrix} 
.\]

{\medskip\noindent\bf Question 4.} It is clearly closed, since by definition the operation always produces a tuple of reals, and since the first entry can never be zero the product can't either. For associativity, let $(a,b),(c,d),(e,f)\in G$. Then $((a,b)(c,d))(e,f)=(a,b)((c,d)(e,f))=(ace, ad+b+acf)$. The inverse of $(a,b)\in G$ is just $(\frac{1}{a}, -\frac{b}{a})$, since $(a,b)(\frac{1}{a},-\frac{b}{a})=(1,0)=I$. Finally for any $(a,b)\in G$ we have $(a,b)(1,0)=(1,0)(a,b)=(a,b)$. Thus $G$ is a group.

{\medskip\noindent\bf Question 7.} If we apply $c$ to both sides of $ab=1$, we get $cab=1\cdot b=b=c$, as required. Since $b=c$ is a left and right inverse of $a$, we have $a^{-1}=b$.

For the forward direction of the second part, let $b=a^{-1}$. Then we have $aba=aa^{-1}a=a$ and $ab^2a=a(a^{-1})^2a=1$ as required. For the backward direction, assume that $aba=a$ and $ab^2a=1$. Then we have that $ab^2$ is a left inverse of $a$ and $b^2a$ is a right inverse, so by the first part of the question $a$ is invertible and we have we have $ab^2=b^2a=a^{-1}$. Applying the inverse:
\[
aba=a\implies ab=1\text{ and }ba=1\implies a^{-1}=b
.\]

{\medskip\noindent\bf Question 8.} Since transformations of the plane can be written as matrices, from linear algebra we have:
\[
    \alpha=\begin{pmatrix} \cos\theta&\sin\theta\\ \sin\theta&-\cos\theta \end{pmatrix}, \rho=\begin{pmatrix} 1&0\\0&-1 \end{pmatrix} 
.\]

Using these to compute the given values:
\[
\rho \alpha\rho^{-1}=\begin{pmatrix} 1&0\\0&-1 \end{pmatrix}\begin{pmatrix} \cos\theta&\sin\theta\\ \sin\theta&-\cos\theta \end{pmatrix}\begin{pmatrix} 1&0\\0&-1 \end{pmatrix}=\begin{pmatrix} \cos\theta&-\sin\theta\\ -\sin\theta&-\cos\theta \end{pmatrix}=\alpha ^{-1}
.\]

{\medskip\noindent\bf Question 11.} In a group every element has an inverse, we have that $\left( ax=b\implies x=a^{-1}b \right) $ and $\left( ya=b\implies y=ba^{-1} \right) $. By closure both of these solutions are themselves in the group, so both equations always have solutions.

For the second part, assume that $ax=b$ and $ya=b$ have solutions for all $a,b\in G$. Let $a,b\in G$. By hypothesis $ax=a$ has a solution $x\in G$. Also $ya=b$ has a solution $y\in G$, and note that $bx=yax=ya=b$, which is true of all $b\in G$. By symmetry there similarly exists $x'\in G$ such that $\forall b\in G, x'b=b$. However these together give that $x'x=x$ and $xx'=x'$. Let $c\in G$, then $(xx') c=xc=x'c=c\implies xc=c$, together with the fact we found previously $cx=x$ giving us $x=1$. 

Since $G$ has a unit, the equations $ax=1$ and $xa=1$ have solutions for all $a\in G$, and applying question 7 to $x$ and $y$ tells us that $a$ is invertible and $x=y=a^{-1}$. Since $G$ is a semigroup that has a unit and each element has an inverse, it is a group.

{\medskip\noindent\bf Question 13.} Let $G$ be a group for which there is no $a\in G$ with $a^2=1$. Note that $a^2=1\implies a=a^{-1}$, so each element is distinct from its inverse except of course the identity. Enumerating all the elements in $G$, each pair $(a,a^{-1})$ adds two to the total, plus the unit for one additional element. However an even number plus an odd one is odd, so $G$ is of odd order. Thus by contrapositive any group of even order has at least one non-1 element that is its own inverse.


\end{document}
