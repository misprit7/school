\documentclass[letterpaper, reqno,11pt]{article}
\usepackage[margin=1.0in]{geometry}
\usepackage{color,latexsym,amsmath,amssymb,graphicx,float,listings,tikz}
\usepackage{hyperref}

\hypersetup{
colorlinks=true,
linkcolor=magenta,
filecolor=magenta,
urlcolor=cyan,
}

\graphicspath{ {images/} }

\begin{document}
\pagenumbering{arabic}
\title{Math 322 Homework 2}
\date{24/09/23}
\author{Xander Naumenko}
\maketitle

{\medskip\noindent\bf Question 1.} Simply using the definition of the maps and manually carrying through where each number gets mapped:

\[
    \alpha\beta=\begin{pmatrix} 1&2&3&4&5\\ 2&1&5&4&3 \end{pmatrix} 
.\]
\[
    \beta\alpha=\begin{pmatrix} 1&2&3&4&5\\ 3&4&1&2&5 \end{pmatrix} 
.\]
\[
    \alpha ^{-1}=\begin{pmatrix} 1&2&3&4&5\\ 3&1&2&5&4 \end{pmatrix} 
.\]

{\medskip\noindent\bf Question 4.} It is clearly closed, since by definition the operation always produces a tuple of reals, and since the first entry can never be zero the product can't either. For associativity, let $(a,b),(c,d),(e,f)\in G$. Then $((a,b)(c,d))(e,f)=(a,b)((c,d)(e,f))=(ace, ad+b+acf)$. The inverse of $(a,b)\in G$ is just $(\frac{1}{a}, -\frac{b}{a})$, since $(a,b)(\frac{1}{a},-\frac{b}{a})=(1,0)=I$. Finally for any $(a,b)\in G$ we have $(a,b)(1,0)=(1,0)(a,b)=(a,b)$. Thus $G$ is a group.

{\medskip\noindent\bf Question 7.} If we apply $c$ to both sides of $ab=1$, we get $cab=1\cdot b=b=c$, as required. Since $b=c$ is a left and right inverse of $a$, we have $a^{-1}=b$.

For the forward direction of the second part, let $b=a^{-1}$. Then we have $aba=aa^{-1}a=a$ and $ab^2a=a(a^{-1})^2a=1$ as required. For the backward direction, assume that $aba=a$ and $ab^2a=1$. 

\end{document}
