\documentclass[letterpaper, reqno,11pt]{article}
\usepackage[margin=1.0in]{geometry}
\usepackage{color,latexsym,amsmath,amssymb,graphicx,float,listings,tikz}
\usepackage{hyperref}

\hypersetup{
colorlinks=true,
linkcolor=magenta,
filecolor=magenta,
urlcolor=cyan,
}

\graphicspath{ {images/} }

\begin{document}
\pagenumbering{arabic}
\title{Math 322 Homework 3}
\date{28/09/23}
\author{Xander Naumenko}
\maketitle

{\medskip\noindent\bf Question 2.} Let $f$ be a map from the rotation $\frac{2\pi}{n}$ to $\cos \frac{2\pi}{n}+i\sin \frac{2\pi}{n}$. The subgroup generated by the following is isomorphic to these groups, as a rotation effectively shifts each point on the $n$-gon or complex unit circle 1 unit in a direction:
\[
    \begin{pmatrix} 1&2&\cdots&n-1&n\\ 2&3&\cdots&n&1
     \end{pmatrix} 
.\]

{\medskip\noindent\bf Question 4.} They are not isomorphic. To see why suppose by contradiction not, i.e. suppose there exists an isomorphism $f:\mathbb{Z}\to \mathbb{Q}$. Let $p=f^{-1}(1)$. Then $f\left( \frac{p}{2} \right) $ is an integer that when added to itself gives $f\left( \frac{p}{2} \right) +f\left( \frac{p}{2} \right) =f\left( \frac{p}{2}+\frac{p}{2} \right) =f(p)=1$, but obviously no such integer exists so $f$ can't actually exist.

{\medskip\noindent\bf Question 5.} They are not isomorphic. Note that $(-1)^{-1}=-1$ in the group of multiplicative group of non-zero rationals, but there is no such element under that additive group of rationals that is it's own inverse. Thus for any bijective map $f:\mathbb{Q}\setminus\{0\}\to\mathbb{Q}$, $f(-1)=f((-1)(-1)(-1))=f(-1)+f(-1)+f(-1)=3f(-1)\implies 3\neq 1$ which is clearly ridiculous, so $f$ can't be an isomorphism.

{\medskip\noindent\bf Question 1.} Simply applying $C$ to the equation $C(C(A))\supset A$, we get that $C(C(C(A)))\subset A$. For the other direction, let $A=C(B)$, then $C(C(C(B)\supset C(B)$. Since this holds for any two sets $A,B$, both sides contain one another and $C(A)=C(C(C(A)))$.

For the second part, let $c\in C(A)$, where $A\subset C(c)$. Then $<A>\subset C(c)$, and $C(A)\subset C(<A>)$ since $<A>$ commutes with everything in $C(A)$.

Finally for the last part, we can use the previous result and the fact that $S\subset C(S)$ (since $S$ commutes) to say $C(S)=C(<S>))=C(M)\implies S\subset C(M)$. Then $M= <S>\subset C(M)\implies$ M is commutative.

{\medskip\noindent\bf Question 3.} Let $G$ be an abelian group, and let $g_1,g_2,\ldots,g_n$ be a finite set of generators in $G$. Then every element $g\in G$ can be expressed as a combination of $g_1^{a_1}g_2^{a_2}\ldots g_n^{a_n},0\leq a_i<o(g_1)$. The number of total possibilities is just the product of the number of choices on each each generator, i.e. $|G|=\prod_i o(g_i)<\infty$.

{\medskip\noindent\bf Question 4.} Let $g$ as described in the question. $(g^{k})^{[n,k]/k}=1$, since $n|[n,k]$ by definition. Thus $o(g^{k})|[n,k]/k$. Also $(g^{k})^{o(g^{k})}=1$ by definition, so $n|ko(g^{k})$. Thus since $ko(g^{k})$ both divides and is divided by $[n,k]$, they must be equal and thus $o(g^{k})=[n,k]/k=n/(n,k)$.

\end{document}
