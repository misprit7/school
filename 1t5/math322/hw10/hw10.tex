\documentclass[letterpaper, reqno,11pt]{article}
\usepackage[margin=1.0in]{geometry}
\usepackage{color,latexsym,amsmath,amssymb,graphicx,float,listings,tikz}
\usepackage{hyperref}

\hypersetup{
colorlinks=true,
linkcolor=magenta,
filecolor=magenta,
urlcolor=cyan,
}

\graphicspath{ {images/} }

\begin{document}
\pagenumbering{arabic}
\title{Math 322 Homework 10}
\date{19/11/23}
\author{Xander Naumenko}
\maketitle

{\medskip\noindent\bf Question 4.6.1.}  The backwards direction is clear, as stated in the textbook (top of page 249) every finite group has a composition series so in particular finite abelian groups do too. For the forward direction, suppose by contradiction that $G$ is an infinite abelian group and $G=G_1\triangleright\ldots\triangleright G_{s+1}=\{1\}$ is a composition series. Since this is a composition series each of the $G_i/G_{i+1}$ are simple, and since they are also abelian they must be of prime order. But then we have:
\[
|G|=[G_1:G_2]|G_2|=\left| G_1 /G_{2} \right| |G_2|=\left| G_1 /G_2 \right| \ldots\left| G_s /G_{s+1} \right| <\infty
.\]
This contradicts our assumption that $G$ was infinite though, so no such composition series can exist.

%Then since $G_1$ is infinite and $G_{s+1}$ is finite with cardinality 1, there must exist a $k\in \mathbb{N}$ with $G_k$ being infinite but $G_{k+1}$ is finite. Let $x\in G_k\setminus G_{k+1}$, and let $H=\langle G_{k+1},x\rangle$ ($G_{k+1}$ is finite so it is finitely generated). Since $G$ is abelian, $G_k$ is as well, and since every subgroup of an abelian group is normal we have $G_k\triangleright H$. Also by construction $H\supset G_{k+1}$, so we have $G_{k}\supset H\supset G_k$

{\medskip\noindent\bf Question 4.6.2.} To show that $p_i$ is a prime, by contradiction assume that it isn't, i.e. assume that $n_i /n_{i+1}=aq$ for some $i\in \mathbb{N}, a,q>1$ and $q$ prime. $G_i /G_{i+1}$ is cyclic and thus abelian, so every subgroup is normal. Also by Cauchy's theorem, there exists a subgroup $H$ of order $q$ in $G_i /G_{i+1}$ since it has order $n_i /n_{i+1}=aq$. But then $G_i\supset HG_{i+1}\supset G_{i+1}$ contradiction our assumption that $G_{i+1}$ is maximal, so in fact $p_i$ was a prime.

Let $n=n_1,n_2,\ldots,n_{s+1}=1$ be a sequence of integers such that $p_i=n_i /n_{i+1}$ is prime. Construct the composition series $G_1=G=\langle a\rangle$, $G_2=\langle a^{n/n_2}\rangle$, $G_3=\langle a^{n /n_3}\rangle, \ldots,G_s=\langle a^{n /n_s}\rangle, G_{s+1}=\{1\}$. Then each $G_i /G_{i+1}$ has order $\frac{n_{i+1}}{n_i}=p_i$ prime, so every subgroup of $G_i /G_{i+1}$ is either the identity or the whole group, which means that $G_{i+1}$ is maximal normal. Also $G_i\triangleright G_{i+1}$ for all $i$, so this is a valid composition series with the required properties.

{\medskip\noindent\bf Question 4.6.3.} Expanding and using the fact that $(g,h)^{-1}=(h,g)$:

\[
    (g,hk)=g^{-1}k^{-1}h^{-1}ghk=(g,k)(gk)^{-1}h^{-1}ghk=(g,k)(g^{-1}h^{-1}gh)^{k}=(g,k)(g,h)^{k}
.\]
\[
    (gh,k)=h^{-1}g^{-1}k^{-1}ghk=h^{-1}g^{-1}k^{-1}g(kh)(h,k)=(g,k)^{h}(h,k)
.\]
\[
    (g^{h},(h,k))(h^{k},(k,g))(k^{g},(g,h))=(g^{h})^{-1}(k,h)g^{h}(h,k)(h^{k})^{-1}(g,k)h^{k}(k,g)(k^{g})^{-1}(h,g)k^{g}(g,h)
\]
\[
=(h^{-1}g^{-1}hk^{-1}h^{-1}kgk^{-1}hk)\left( k^{-1}h^{-1}kg^{-1}k^{-1}ghg^{-1}kg \right) \left( g^{-1}k^{-1}gh^{-1}g^{-1}hkh^{-1}gh \right)=1
.\]

{\medskip\noindent\bf Question 4.6.4.} Let $h\in H,k\in K$. Then since $(K,H)$ is a group $(k,h)^{-1}(k^{-1}h^{-1}kh)^{-1}=h^{-1}k^{-1}hk=(h,k)\in (K,H)$. Since $h,k$ were arbitrary we have that $(K,H)$ contains all the generators of $(H,K)$, so $(H,K)\subseteq(K,H)$. By symmetry the exact same argument works in reverse to show $(H,K)\supseteq (K,H)$, so $(H,K)=(K,H)$.

To show normality, let $g\in G$ and $x\in (H,K)$. Since $(H,K)$ is generated by commutators $(h,k)$ it can be broken up as $x=(h_1,k_1)\cdots (h_n,k_n)$. Then we have
\[
g^{-1}xg=(g^{-1}(h_1,k_1)g)(g^{-1}(h_2,k_2)g)\cdots (g^{-1}(h_n,k_n)g)
.\]
Thus it suffices to show that $g^{-1}(h,k)g\in (H,K)\forall h\in H,k\in K$. Since $H$ and $K$ are normal we have $g^{-1}hg\in H,g^{-1}kg\in K$. Then:
\[
g^{-1}(h,k)g=g^{-1}h^{-1}gg^{-1}k^{-1}gg^{-1}hgg^{-1}kg=(g^{-1}hg)^{-1}(g^{-1}kg)^{-1}(g^{-1}hg)(g^{-1}kg)\in (H,K)
.\]
Since all the generators are expressible in terms of one another $(H,K)=(K,H)$ are normal in $G$.

{\medskip\noindent\bf Herstein, Question 2.13.11a.} There is exactly one subgroup of order $q$, since Sylow II says that $n_q$ (the number of $q$-Sylow subgroups) divides the index of $q$-Sylow subgroups which is $p$, but since $p<q$ this forces $n_q=1$. As for $p$, Sylow II forces $n_p\equiv 1\mod p\implies n_p=pk+1,k\in \mathbb{N}\cup \{0\}$. But then $pk+1|q$, and since $q$ is a prime either $pk+1=q$ or $k=0$. The first case can't happen though since then $pk=q-1\implies p|q-1$ which it doesn't by hypothesis, so $n_p=p(0)+1=1$ as well.

Therefore there are unique $p,q$-Sylow subgroups. Since $p<q$ it can't be that $p=q=2$, so $p+q<pq$. Therefore there's at least one element $g\in G$ s.t. $g$ isn't in either the $p$-Sylow or $q$-Sylow subgroup. $g$ can't have order $1$ since then it would be the identity and be in both, it can't have order $p$ since then $\langle g\rangle$ would be a $p$-Sylow subgroup (and $g$ was chosen not to be in any), and similarly it can't have order $q$ since then $\langle g\rangle$ would be a $q$-Sylow subgroup. Therefore the only possibility is that $|g|=pq$, so $G=\langle g\rangle$ is cyclic.


\end{document}
