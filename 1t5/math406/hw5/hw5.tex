\documentclass[letterpaper, reqno,11pt]{article}
\usepackage[margin=1.0in]{geometry}
\usepackage{color,latexsym,amsmath,amssymb,graphicx,float,listings,tikz}
\usepackage{hyperref}

\hypersetup{
colorlinks=true,
linkcolor=magenta,
filecolor=magenta,
urlcolor=cyan,
}

\lstset{
  basicstyle=\ttfamily,
  columns=fullflexible,
  frame=single,
  breaklines=true,
  postbreak=\mbox{\textcolor{red}{$\hookrightarrow$}\space},
}

\graphicspath{ {images/} }

\begin{document}
\pagenumbering{arabic}
\title{Math 406 Homework 5}
\date{17/11/23}
\author{Xander Naumenko}
\maketitle

{\medskip\noindent\bf Question 1a.} For all functions $v$ in some class, the following must be true:
\[
\int_{0}^{1}v(u''+k^2u-f)dx=0
.\]
Integrating by parts:
\[
\int_0^{1}-u'v'+k^2uv-fvdx+v(1)\beta-v(0)u'(0)=0\implies \int_0^{1}u'v'dx=k^2\int_0^{1}uvdx-\int_0^{1}fvdx+\beta v(1)
.\]
Thus the weak form is to find $u\in H_\alpha^{1}$ such that $\int_0^{1}u'v'dx=k^2\int_0^{1}uvdx-\int_0^{1}fvdx+\beta v(1)\forall v\in H_0^{1}$.

{\medskip\noindent\bf Question 1b.} Since we used the strong form to derive the weak form while relaxing constraints, the strong form always implies the weak form. 

{\medskip\noindent\bf Question 1c.} Let $u^{h}(x)=\sum_{n=0}^{N}u_nN_n(x)$, where $N_n(x)$ are the basis functions. Similarly let $v^{h}(x)=\sum_{n=1}^{N}v_nN_n(x)$. Using the weak form derived above, we need to find $u^{h}\in V_\alpha^{h}$ such that:
\[
\int_0^{1}\left(\alpha N_0'+\sum_{n=1}^{N}u_nN_n'\right)\left( \sum_{m=1}^{N}v_mN_m' \right) dx=k^2\int_0^{1}\left( \alpha N_0+\sum_{n=1}^{N}u_nN_n \right) \left( \sum_{m=1}^{N}v_mN_m \right) dx
\]
\[
-\int_0^{1}f\left( \sum_{m=1}^{N}v_mN_m \right) dx+\sum_{m=1}^{N}v_m N_m(1)\beta=0
.\]
\[
    \implies\sum_{m=1}^{N}v_m\bigg[\alpha\int_0^{1}N_0' N_m'dx+\sum_{n=1}^{N}u_n\int_0^{1}N_n' N_m'dx-k^2\left( \alpha\int_0^{1} N_0 N_m dx+\sum_{n=1}^{N}u_n\int_0^{1}N_n N_m dx \right) 
\]
\[
+\int_0^{1}fN_mdx-N_m(1)\beta\bigg]=0
.\]
\[
\implies\sum_{n=1}^{N}u_n(K_{mn}-k^2M_{mn})=\delta_{mn}\beta-\alpha(K_{0m}-k^2M_{0m})+\int_0^{1}fN_mdx\implies (K-k^2M)=b
.\]
Here $K$ is the stiffness matrix and $M$ is the mass matrix. This derivation was done in class and in the notes so most of the algebra is taken from there.

{\medskip\noindent\bf Question 1d.} To use piecewise linear basis functions with discretization size $h$, all we have to do is construct the stiffness and mass matrices. Consider just a single element $e$ along with the change of coordinates $x(\xi)=x_{e-1}N_1(\xi)+x_eN_2(\xi)$ where $N_1$ and $N_2$ are the two linear basis functions in $[-1,1]$ (with corresponding $N_a(x),N_b(x)$ in the $x$ domain):
\[
    \int_{x_{e-1}}^{x_e}N_m'(x)N_n'(x)dx=\int_{-1}^{1}\frac{dN_a}{d\xi}\frac{d\xi}{dx}\frac{dN_b}{d\xi}\frac{d\xi}{dx}\frac{dx}{d\xi}d\xi=\frac{2}{h}\int_{-1}^{1}\frac{\xi_a\xi_b}{4}d\xi=\frac{\xi_a\xi_b}{h}
\]
\[
    \implies K_{ab}^{e}=\frac{1}{h} \begin{pmatrix} 1&-1\\-1&1 \end{pmatrix} 
.\]
Similarly for the mass matrix:
\[
M_{mn}^{e}=\int_{x_{e-1}}^{x_e}N_mN_ndx=\int_{-1}^{1}\frac{1}{4}\left( 1+\xi_a\xi \right)\left( 1+\xi_b\xi \right) \frac{dx}{d\xi}d\xi=\frac{h}{4}\left( 1+\frac{\xi_a\xi_b}{3} \right) 
\]
\[
    \implies M^{e}=\frac{h}{6}\begin{pmatrix} 2&1\\1&2 \end{pmatrix} 
.\]
Putting these together and summing over all the elements (and being careful about summing the correct conditions at the boundary, we get the final matrices to be:
\[
    K=\frac{1}{h}\begin{pmatrix} 2&-1&0&0&0&\cdots\\-1&2&-1&0&0&\cdots\\ 0&0&-1&2&-1&\cdots\\ \vdots\\ 0&0&0&0&\cdots&-1&2&-1\\ 0&0&0&0&\cdots&0&-1&1\end{pmatrix} 
\]
\[
    M=\frac{h}{6}\begin{pmatrix} 4&1&0&0&0&\cdots\\1&4&1&0&0&\cdots\\ 0&0&1&4&1&\cdots\\ \vdots\\ 0&0&0&0&\cdots&1&4&1\\ 0&0&0&0&\cdots&0&1&2\end{pmatrix} 
.\]
Finally we have also have:
\[
b = \begin{pmatrix} -\left(\frac{1}{h}-\frac{k^2h}{6}\right)\alpha-(f,N_1)\\-(f,N_2)\\-(f,N_3)\\\vdots \\\beta-(f,N_N) \end{pmatrix} \text{ where } (f,N_m)=\int_{0}^{1}fN_mdx
.\]
Thus the finite element discretization is to solve the equation $(K-k^2M)u=b$ with $K,M$ and $b$ defined as above.

{\medskip\noindent\bf Question 1e.} The solution to the homogeneous equation $u''+10^2u=0$ is $u=A\sin 10x+B\cos 10x$. For the nonhomogeneous equation assume that $u=a_1+a_2x+a_3x^2+a_4x^{3}$, then we have $a_3+a_4x+100a_1+100a_2x+100a_3x^2+100a_4x^{3}=x^{3}\implies u=-\frac{6x}{10^{4}}+\frac{x^{3}}{10^2}$. Thus plugging in boundary conditions, the analytic solution is $u(x)=\frac{\sin 10x}{10\cos 10}\left(1+\frac{6}{10^{4}}-\frac{3}{10^2}\right)-\frac{6x}{10^{4}}+\frac{x^{3}}{10^2}$.

As for the numerical solution, the matrices described above were construction for $N=10,20,30$ and solved. The results can be seen in figure .

{\medskip\noindent\bf Question 1f.} All of the work for the previous parts have been to reduce the problem to one in the given form. Here $f=\alpha=\beta=0$ so $b=0$, and thus $Ku=k^2Mu$. Letting $x=u, A=K,B=M$ and $\lambda=k^2$, we see that this is the desired eigenvalue problem.

{\medskip\noindent\bf Question 2a.} From the main boundary value problem, we have that $\frac{D}{r}(rp_r)_r=0$, which has solutions in the form $p(r)=A+B\log r$. The boundary conditions forces $\lim_{r\to 0}r \frac{\partial p}{\partial r}=B=-\frac{Q_0}{2\pi D}$, and $u(R(t),t)=0\implies A=\frac{Q_0}{2\pi D}\log R(t)$. Putting this together we have $p(r,t)=\frac{Q_0}{2\pi D}\log \frac{R(t)}{r}$.

Using the Stefan condition for front velocity:
\[
\dot R(t)= -\frac{w_0^2}{\mu'}\frac{dp}{dr}\bigg]_{r=R(t)}=\frac{w_0^2}{\mu'}\frac{Q_0}{2\pi DR(t)}
.\]
Finally using separation of variables we can solve this differential equation:
\[
\int R(t)dR(t)=\int_0^{t}\frac{w_0^2}{\mu'}\frac{Q_0}{2\pi D}dt\implies R(t)=\sqrt{\frac{w_0^2Q_0t}{\mu'\pi D}}
.\]

{\medskip\noindent\bf Question 2bi.} Expanding, the equation to solve is $Lp=p_{rr}+\frac{1}{r}p_r=g(r,t)$. We can find the adjoint operator:
\[
\int_0^{R(t)}Gfrdr=\int_0^{R(t)}G\cdot (Lu)\cdot rdr=\int_0^{R(t)}G\left( ru''+u' \right) dr
.\]
From this point the self adjoint operator and associated boundary terms were derived in class (lecture 14), they are:
\[
  =\left[Ga_0u'+Ga_1u-(a_0G)'u\right]_{0}^{R(t)}+\int_{0}^{R(t)}u\left( (a_0G)''-(a_1G)'+a_2G \right) dr
\]
\[
=-+\int_0^{R(t)}u\left( rG''+G' \right) dr
.\]
Thus under the radial integration $L$ acts as self adjoint (there's probably some deeper reason for the way the algebra works out with the extra $r$, but from the algebra it works out at least). Thus we want to find $G(s,r)$ such that $sG_{ss}(s,r)+G_s(s,r)=\delta(s-r)$. We already determined that the solution to the homogeneous equation is $p(r)=A+B\log r$. Thus we can split the Green's function into two parts:
\[
G(s,r)=\begin{cases}
  A_-+B_-\log s&0<s<r\\
  A_++B_+\log s&r<s<R(t)
\end{cases}
.\]
The boundary conditions force $B_-= -\frac{Q_0}{2\pi D}$ and $A_+=-B_+\log R(t)$. Continuity forces that $A_--\frac{Q_0}{2\pi D}\log r=B_+\log \frac{r}{R(t)}\implies A_-=B_+\log \frac{r}{R(t)}+\frac{Q_0}{2 \pi D}\log r$. Thus we can write $G$ once again as
\[
G(s,r)=\begin{cases}
  B_+\log \frac{r}{R(t)}+\frac{Q_0}{2\pi D}\log \frac{r}{s}&0<s<r\\
  B_+\log \frac{s}{R(t)}&r<s<R(t)
\end{cases}
\]
\[
G_s(s,r)=\begin{cases}
  -\frac{Q_0}{2\pi Ds}&0<s<r\\
  \frac{B_+}{s}&r<s<R(t)
\end{cases}
.\]
The only condition left is the jump condition:
\[
\int_{r-\epsilon}^{r+\epsilon}sG_{ss}+G_sds=\int_{r-\epsilon}^{r+\epsilon}(sG_s)_sds=sG_s\bigg|_{r-\epsilon}^{r+\epsilon}=r\left( G_s(r_+,r)-G_s(r_-,r) \right)=\int_{r-\epsilon}^{r+\epsilon}s\delta(r-s)ds=r
\]
\[
\implies 1=\frac{1}{r}B_++\frac{Q_0}{2\pi Dr}\implies B_+=r-\frac{Q_0}{2\pi D}
.\]
We can then get a final formulation for $G$:
\[
G(s,r)=\begin{cases}
  r\log \frac{r}{R(t)}+\frac{Q_0}{2\pi D}\log \frac{R(t)}{s}&0<s<r\\
  \left( r-\frac{Q_0}{2\pi D} \right) \log \frac{s}{R(t)}&r<s<R(t)
\end{cases}=-\frac{Q_0}{2\pi D}\log \frac{s}{R(t)}+\begin{cases}
  r\log \frac{r}{R(t)}&0<s<r\\
  r\log \frac{s}{R(t)}&r<s<R(t)
\end{cases}
.\]
Putting this together we can determine an expression for $p(r,t)$ in terms of $R(t)$ and $g(r,t)$:
\[
p(r,t)=\int_0^{r}r\log \frac{r}{R(t)}g(s)sds+\int_r^{R(t)}r\log \frac{s}{R(t)}g(s)sds-\int_0^{R(t)}\frac{Q_0}{2\pi D}\log \frac{s}{R(t)} g(s)sds
.\]

{\medskip\noindent\bf Question 2bii.} Plugging in $r=R(t)$:
\[
  p(R(t),t)=
.\]

\end{document}
