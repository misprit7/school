\documentclass[letterpaper, reqno,11pt]{article}
\usepackage[margin=1.0in]{geometry}
\usepackage{color,latexsym,amsmath,amssymb,graphicx,float,listings,tikz}
\usepackage{hyperref}

\hypersetup{
colorlinks=true,
linkcolor=magenta,
filecolor=magenta,
urlcolor=cyan,
}

\graphicspath{ {images/} }

\begin{document}
\pagenumbering{Arabic}
\title{Math 320 Homework 2}
\date{19/09/23}
\author{Xander Naumenko}
\maketitle

{\medskip\noindent\bf Question 1.} Proof by strong induction.

{\bf Base Case ($n=1$ and $n=2$):} Explicitly calculating:
\[
F(1)=\frac{\varphi^{2}-(-\varphi)^{-2}}{\sqrt{5}}=\frac{1}{\sqrt{5} }\left( \varphi+1 -(\varphi+1)^{-1}\right)=\frac{1}{\sqrt{5} }\left( \frac{3+\sqrt{5} }{2}-\frac{2}{3+\sqrt{5} } \right)=\frac{10+6\sqrt{5} }{10+6\sqrt{5} }=1
.\]
\[
F(2)=\frac{\varphi^{3}-(-\varphi)^{-3}}{\sqrt{5}}=\frac{1}{\sqrt{5} }\left( \varphi(\varphi+1) -(\varphi(\varphi+1))^{-1}\right)=\frac{1}{\sqrt{5} }\left( \frac{(3+\sqrt{5})(1+\sqrt{5}  }{4}-\frac{4}{(3+\sqrt{5})(1+\sqrt{5} ) } \right)=2
.\]

{\bf Inductive step:} Assume the result holds for all numbers smaller than $n$, in particular $n-2$ and $n-1$. Then applying the fact $\varphi^2=\varphi+1$ we have:
\[
F(n)=F(n-1)+F(n-2)=\frac{1}{\sqrt{5} }\left( \varphi^{n}-(-\varphi)^{-n}+\varphi^{n-1}-(-\varphi)^{-(n-1)} \right) 
\]
\[
=\frac{1}{\sqrt{5} }\left( \varphi^{n+1}-\varphi^{n-1}-(-\varphi)^{-n}+\varphi^{n-1}-(-\varphi)^{-n-1}+(-\varphi)^{-n} \right) =\frac{1}{\sqrt{5} }\left( \varphi^{n+1}-(-\varphi)^{-n-1} \right) 
.\]

Thus by induction the result holds for all $n\geq 1$.

\newpage

{\medskip\noindent\bf Question 2.} Consider listing the natural numbers in diagonal lines in subsequent order just as discussed in class. For example:

{\noindent
1\ 3\ \ 6\ \ 10\\
2\ 5\ \ 9\ \ 14\\
4\ 8\ 13\ \ 19\\
7\ 12\ 18\ 25\\
}

Let $s^{(k)}$ be the entries of the $k$th row. Clearly they are increasing since for a given row entries are inserted from left to right with numbers that are increasing. Explicitly writing out the sequences, we can just use the values shown above:
\[
s^{(1)}=\left( 1, 3, 6, 10 \right) 
\]
\[
s^{(2)}=\left( 2, 5, 9, 14 \right) 
\]
\[
s^{(3)}=\left( 4, 8, 13, 19 \right) 
\]
\[
s^{(4)}=\left( 7, 12, 18, 25 \right) 
.\]

\newpage

{\medskip\noindent\bf Question 3.} $\mathcal S$ is uncountable. We proved in class that $2^{\mathbb N}$ (the power set) is uncountable. For any $a\in 2^{\mathbb N}$, let $f:2^{\mathbb N}\to \mathcal S$ with $f(a)$ denoting the sequence obtained by considering $a$ as an increase sequence. This is possible since $\mathbb N$ is countable and well ordered, so such a sequence can always be generated by repeatedly taking the minimum value of the set and adding it to the end of a sequence. $f$ is onto, since for any $s\in\mathcal S$, the set of its elements is by definition in $2^{\mathbb N}$. $f$ is also 1-1, since for any $a, b\in 2^{\mathbb N}$, $f(a)=f(b)$ would imply that $a$ and $b$ share all elements which is the same as $a=b$. Thus $f$ is a bijection and $|2^{\mathbb N}|=|\mathcal S|$, so $\mathcal S$ is uncountable.

\newpage

{\medskip\noindent\bf Question 4.} 

\newpage

{\medskip\noindent\bf Question 6a.} Note that there is a bijection from polynomials with integer coefficients and $\mathbb{Z}^{n}$, where $n$ is the degree of the polynomial. For each $n\in \mathbb{N}\cup \{0\} $, define $P_n$ as the set of all integer coefficient polynomials with degree $n$. Because of the aforementioned bijection $P_n$ is countable for all $n$. Then the set of all polynomials with integer coefficients is just
\[
    P=\bigcup_{n} P_n
,\]
which is a union of finitely many countable sets which we proved in class is itself countable.

{\medskip\noindent\bf Question 6b.} Note that each $P_n$ in the previous part has at most $n$ zeros. Let $Z_n$ be the zeros of all polynomials in $P_n$, since $P_n$ is countable then $Z_n$ is as well. Thus the set of all zeros resulting from integer valued coefficients is just
\[
    Z=\bigcup_{n} Z_n
,\]
which again is the union of countably many countable sets so is countable.

{\medskip\noindent\bf Question 6c.} We proved that $Z$, the set of all algebraic numbers, is countable. Let $f:Z\to \mathbb{R}$ with $f(x)=x$, since $|Z|<|\mathbb{R}|$, $f$ is not onto. Thus $\exists x\in \mathbb{R}$ with $x\notin Z$, i.e. $x\in \mathbb{R}\setminus Z$ as required.

{\medskip\noindent\bf Question 6d.} The set of transcendental numbers is uncountable, to show why assume not, so assume it is either finite or countable. If it was finite, then you could repeat the process taken in part c to find an element in $\mathbb{R}$ not in $Z$, remove it from $\mathbb{R}$, and within finitely many steps you would end up with $\mathbb{R}$ being countable. Since removing a single element at a time doesn't change countability, this can't be the case and there can't be finitely many transcendental numbers.

Let $T$ be the set of transcendental numbers. By definition $\mathbb{R}=Z\cup T$. If $T$ was countable, then $Z\cup T$ is the union of countable sets, so $\mathbb{R}$ would also be countable. Since this isn't true, our assumption was wrong and $T$ was uncountable.

\newpage

{\medskip\noindent\bf Question 7a.} 

\end{document}
