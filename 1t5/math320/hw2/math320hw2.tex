\documentclass[letterpaper, reqno,11pt]{article}
\usepackage[margin=1.0in]{geometry}
\usepackage{color,latexsym,amsmath,amssymb,graphicx,float,listings,tikz}
\usepackage{hyperref}

\hypersetup{
colorlinks=true,
linkcolor=magenta,
filecolor=magenta,
urlcolor=cyan,
}

\graphicspath{ {images/} }

\begin{document}
\pagenumbering{arabic}
\title{Math 320 Homework 2}
\date{19/09/23}
\author{Xander Naumenko}
\maketitle

{\medskip\noindent\bf Question 1.} Proof by strong induction.

{\bf Base Case ($n=1$ and $n=2$):} Explicitly calculating:
\[
F(1)=\frac{\varphi^{2}-(-\varphi)^{-2}}{\sqrt{5}}=\frac{1}{\sqrt{5} }\left( \varphi+1 -(\varphi+1)^{-1}\right)=\frac{1}{\sqrt{5} }\left( \frac{3+\sqrt{5} }{2}-\frac{2}{3+\sqrt{5} } \right)=\frac{10+6\sqrt{5} }{10+6\sqrt{5} }=1
.\]
\[
F(2)=\frac{\varphi^{3}-(-\varphi)^{-3}}{\sqrt{5}}=\frac{1}{\sqrt{5} }\left( \varphi(\varphi+1) -(\varphi(\varphi+1))^{-1}\right)=\frac{1}{\sqrt{5} }\left( \frac{(3+\sqrt{5})(1+\sqrt{5}  }{4}-\frac{4}{(3+\sqrt{5})(1+\sqrt{5} ) } \right)=2
.\]

{\bf Inductive step:} Assume the result holds for all numbers smaller than $n$, in particular $n-2$ and $n-1$. Then applying the fact $\varphi^2=\varphi+1$ we have:
\[
F(n)=F(n-1)+F(n-2)=\frac{1}{\sqrt{5} }\left( \varphi^{n}-(-\varphi)^{-n}+\varphi^{n-1}-(-\varphi)^{-(n-1)} \right) 
\]
\[
=\frac{1}{\sqrt{5} }\left( \varphi^{n+1}-\varphi^{n-1}-(-\varphi)^{-n}+\varphi^{n-1}-(-\varphi)^{-n-1}+(-\varphi)^{-n} \right) =\frac{1}{\sqrt{5} }\left( \varphi^{n+1}-(-\varphi)^{-n-1} \right) 
.\]

Thus by induction the result holds for all $n\geq 1$.

\newpage

{\medskip\noindent\bf Question 2.} Consider listing the natural numbers in diagonal lines in subsequent order just as discussed in class. For example:

{\noindent
1\ 3\ \ 6\ \ 10\\
2\ 5\ \ 9\ \ 14\\
4\ 8\ 13\ \ 19\\
7\ 12\ 18\ 25\\
}

Let $s^{(k)}$ be the entries of the $k$th row. Clearly they are increasing since for a given row entries are inserted from left to right with numbers that are increasing. Explicitly writing out the sequences, we can just use the values shown above:
\[
s^{(1)}=\left( 1, 3, 6, 10 \right) 
\]
\[
s^{(2)}=\left( 2, 5, 9, 14 \right) 
\]
\[
s^{(3)}=\left( 4, 8, 13, 19 \right) 
\]
\[
s^{(4)}=\left( 7, 12, 18, 25 \right) 
.\]

\newpage

{\medskip\noindent\bf Question 3.} $\mathcal S$ is uncountable. We proved in class that $2^{\mathbb N}$ (the power set) is uncountable. For any $a\in 2^{\mathbb N}$, let $f:2^{\mathbb N}\to \mathcal S$ with $f(a)$ denoting the sequence obtained by considering $a$ as an increase sequence. This is possible since $\mathbb N$ is countable and well ordered, so such a sequence can always be generated by repeatedly taking the minimum value of the set and adding it to the end of a sequence. $f$ is onto, since for any $s\in\mathcal S$, the set of its elements is by definition in $2^{\mathbb N}$. $f$ is also 1-1, since for any $a, b\in 2^{\mathbb N}$, $f(a)=f(b)$ would imply that $a$ and $b$ share all elements which is the same as $a=b$. Thus $f$ is a bijection and $|2^{\mathbb N}|=|\mathcal S|$, so $\mathcal S$ is uncountable.

\newpage

{\medskip\noindent\bf Question 4.} Since the rationals are countable, there exists a way of listing them in the interval $[0,1]$ as $\{x:x\in\mathbb{Q}, 0\leq x\leq 1\}=\{q_1, q_2, q_3, \ldots\}$. Without loss of generality choose such a listing with $q_1=0$, $q_2=1$. Define a function $f:[0,1]\to(0,1)$ as
\[
f(x)=\begin{cases}
    q_{i+2}&\text{ if } x=q_i\in \mathbb{Q}\\
    x&\text{ otherwise}
\end{cases}
.\]

$f$ is injective, since for irrationals the identity is injective, and for rationals the list $q_1, q_2, q_3, \ldots$ had each element listed only once. By our choice of $q_1=0$ and $q_2=1$, we have that $0,1\notin f([0,1])$. To show $f$ is onto, let $y\in (0,1)$. If $y\notin \mathbb{Q}$, $f(y)=y$ and we're done. Otherwise $y=q_i$ for some $i>2$ ($i$ can't be 0 or 1 since $y\in (0,1)$), and we have that $f(q_{i-2})=q_i=y$. Thus $f$ is bijective as required.

\newpage

{\medskip\noindent\bf Question 5.} As the hint states, note that $\sqrt{2} $ is irrational. Note that $\forall n\in \mathbb{N}$, $n\sqrt{2} $ is also irrational, since if it could be written as $n\sqrt{2} =\frac{a}{b}$ for some $a\in \mathbb{Z}, b\in \mathbb{N}$, then $\sqrt{2} =\frac{a}{bn}$ which contradicts $\sqrt{2} $'s irrationality. Define a function $f:\mathbb{R}\to \mathbb{R}\setminus \mathbb{Q}$ as follows:
\[
f(x)=\begin{cases}
    (2n-1)\sqrt{2} &\text{ if }x=q_n\in \mathbb{Q}, n\in \mathbb{N}\\
    2m\sqrt{2} &\text{ if }x=m\sqrt{2}, m\in \mathbb{N}\\
    x&\text{ otherwise}
\end{cases}
.\]
$f$ is well defined, since all $x$ part of the second case are irrational and thus disjoint from the first case. $f$ is claimed to be the desired bijection. For injectivity, let $x,y\in \mathbb{R}$ with $f(x)=f(y)$. If $f(x)\neq n\sqrt{2} $ for any $n\in \mathbb{N}$, then $x=f(x)=f(y)=y\implies x=y$. If $f(x)=n\sqrt{2}, n\in \mathbb{N}$, then $x$ must be in one of the first two cases, and since they're disjoint either $x=y=q_{\frac{n+1}{2}}$ or $x=y=\frac{n}{2}\sqrt{2} $.

To show that $f$ is surjective, let $y\in \mathbb{R}\setminus \mathbb{Q}$. If $y\neq n\sqrt{2}$ for any $n\in \mathbb{N}$, then $f(y)=y$. If there exists an $m$ such that $y=2m\sqrt{2} $, then $f(m\sqrt{2})=2m\sqrt{2} =y$. Finally, if there exists an $n\in \mathbb{N}$ with $y=(2n-1)\sqrt{2} $, then $f(q_n)=(2n-1)\sqrt{2} =y$. Since in all cases there exists an $x\in \mathbb{R}$ with $f(x)=y$, $f$ is injective. Thus $f$ is a bijection from $\mathbb{R}$ to $\mathbb{R}\setminus \mathbb{Q}$ as required. For the values requested:
\[
f(\pi)=3.141593
\]
\[
f(\sqrt{3} )=1.732051
\]
\[
f(q_2)=4.242641
.\]
\[
f(q_3)=7.071068
.\]

\newpage

{\medskip\noindent\bf Question 6a.} Note that there is a bijection from polynomials with integer coefficients and $\mathbb{Z}^{n}$, where $n$ is the degree of the polynomial. For each $n\in \mathbb{N}\cup \{0\} $, define $P_n$ as the set of all integer coefficient polynomials with degree $n$. Because of the aforementioned bijection $P_n$ is countable for all $n$. Then the set of all polynomials with integer coefficients is just
\[
    P=\bigcup_{n} P_n
,\]
which is a union of finitely many countable sets which we proved in class is itself countable.

{\medskip\noindent\bf Question 6b.} Note that each $P_n$ in the previous part has at most $n$ zeros. Let $Z_n$ be the zeros of all polynomials in $P_n$, since $P_n$ is countable then $Z_n$ is as well. Thus the set of all zeros resulting from integer valued coefficients is just
\[
    Z=\bigcup_{n} Z_n
,\]
which again is the union of countably many countable sets so is countable.

{\medskip\noindent\bf Question 6c.} We proved that $Z$, the set of all algebraic numbers, is countable. Let $f:Z\to \mathbb{R}$ with $f(x)=x$, since $|Z|<|\mathbb{R}|$, $f$ is not onto. Thus $\exists x\in \mathbb{R}$ with $x\notin Z$, i.e. $x\in \mathbb{R}\setminus Z$ as required.

{\medskip\noindent\bf Question 6d.} The set of transcendental numbers is uncountable, to show why assume not, so assume it is either finite or countable. If it was finite, then you could repeat the process taken in part c to find an element in $\mathbb{R}$ not in $Z$, remove it from $\mathbb{R}$, and within finitely many steps you would end up with $\mathbb{R}$ being countable. Since removing a single element at a time doesn't change countability, this can't be the case and there can't be finitely many transcendental numbers.

Let $T$ be the set of transcendental numbers. By definition $\mathbb{R}=Z\cup T$. If $T$ was countable, then $Z\cup T$ is the union of countable sets, so $\mathbb{R}$ would also be countable. Since this isn't true, our assumption was wrong and $T$ was uncountable.

\newpage

{\medskip\noindent\bf Question 7a.} Since $|A|=|C|$ and $|B|=|D|$, there exists bijections $f:A\to C,g:B\to D$. Let $h:A\cup B\to C\cup D$ be a function defined as
\[
h(x)=\begin{cases}
    f(x)&\text{ if }x\in A\\
    g(x)&\text{ if } x\in B
\end{cases}
.\]
Since $A\cap B=\emptyset$ this is well defined. $h$ is onto, since for every $y\in C\cup D$, either $f^{-1}(y)$ or $g^{-1}(y)$ is in $A\cap B$ which fed back into $h$ gives $y$. Similarly $h$ is injective, since if it wasn't it would imply one of $f,g$ wasn't injective. Thus $h$ is a bijection and $|A|+|B|=|C|+|D|$.

{\medskip\noindent\bf Question 7b.} Let $A_t=\{(x, a)\in \mathbb{R}\times A: x=t\}$. For every $t\in \mathbb{R}$, consider the function $f:A_t\to A$ defined as $f((x, a))=a$. This is clearly onto by definition, and since for a given $A_t$, $(x,a)\in A_t\implies x=t$, it is also injective. Thus it is a bijection as required. The sets $A_t$ are all mutually disjoint since they all have different values for the first entry of the pair.

\medskip

For the following parts I've stated the bijection and if it's obvious I don't explicitly justify 1-1/onto, I don't particularly feel like copying the same boilerplate and I'd appreciate not getting docked marks for not stating the obvious a bunch of times.

{\medskip\noindent\bf Question 7c.} Let $S_n=\{s_1, s_2, \ldots, s_n\}$ be a set with $|S_n|=n$. Then define a function $f:\mathbb{N}\to S_n\cup \mathbb{N}$ as:
\[
f(k)=\begin{cases}
    s_k&\text{ if } k\leq n\\
    k-n&\text{ otherwise}
\end{cases}
.\]

$f$ is is clearly a bijection, since every $s_k$ and natural number is covered as outputs and each input results in a distinct element. Thus $n+\aleph_0=\aleph_0$.

{\medskip\noindent\bf Question 7d.} Let $A=\{a_1, a_2, \ldots\}, B=\{b_1, b_2, \ldots\}$ be two countable sets. Define $f:\mathbb{N}\to A\cup B$ be defined as:
\[
f(n)=\begin{cases}
    a_{n /2}&\text{ if } x\text{ even}\\
    b_{(n+1) /2}&\text{ if }x\text{ odd}
\end{cases}
.\]
Every element is covered in this process uniquely so $f$ is a bijection, so $\aleph_0+\aleph_0=\aleph_0$

{\medskip\noindent\bf Question 7e.} Since the rationals are countable, we can write them as $\mathbb{Q}=\{q_1, q_2, \ldots\}$. Let $A=\{a_1, a_2, \ldots\}$ be a countable set. Define $f:\mathbb{R}\to (\mathbb{R}\setminus \mathbb{Q})\cup A$ to be
\[
f(x)=\begin{cases}
    a_n&\text{ if }x=q_n\in \mathbb{Q}\\
    x&\text{ otherwise}
\end{cases}
.\]
We proved in question 5 that $|\mathbb{R}|=|\mathbb{R}\setminus \mathbb{Q}|$, so $f$ is a bijection between sets of cardinality $c$ and $c+\aleph_0$, so $\aleph_0+c=c$.

{\medskip\noindent\bf Question 7f.} Let $A=[-1, 0]$, $B=(0, 1]$ and $C=[-1, 1]$. We proved in class that $|A|=|B|=|C|=c$. However the function $f:A\cup B\to C$ defined simply as $f(x)=x$ is clearly bijective and shows that $c+c=c$.

\newpage

{\medskip\noindent\bf Question 8a.} I claim $A_1=\{\frac{1}{2}\}=\bigcap \mathcal A$. $[\frac{1}{2}, \frac{1}{2}]=\{\frac{1}{2}\}\in\mathcal A$, so $A_1\supseteq \bigcap \mathcal A$. For any $n\geq 2$, $[\frac{1}{n}, 1-\frac{1}{n}]\subseteq [\frac{1}{2}, 1-\frac{1}{2}]=\{\frac{1}{2}\}$, so $\frac{1}{2}\in A_1$ and thus $A_1\subseteq \bigcap\mathcal A$.

I claim $A_2=(0, 1)=\bigcup\mathcal A$. Let $x\in (0, 1)$. Let $d=\min\left( x, 1-x \right) $. By the Archimedean property, $\exists n\in \mathbb{N}$ s.t. $n> \frac{1}{d}\implies \frac{1}{n}< d$. Thus $[\frac{1}{n}, 1-\frac{1}{n}]\supset [d, 1-d]\supset \{x\}$, so $x\in A_2$ and thus $A_2\subseteq \bigcup \mathcal A$. For $y\in \mathbb{R}\setminus (0,1)$, let $n\in \mathbb{N}, n\geq 2$. $y\notin[\frac{1}{n}, 1-\frac{1}{n}]$, so $A_2\supseteq \bigcup\mathcal A$.

{\medskip\noindent\bf Question 8b.} I claim $B_1=[-1, 1]=\bigcap\mathcal A$. Let $x\in [-1, 1]$. For any $n\in \mathbb{N}$, $x\in\left(-1-\frac{1}{n}, 1+\frac{1}{n}\right)$, so $B_1\subseteq \bigcap\mathcal A$. For $y\in \mathbb{R}\setminus [-1, 1]$, let $d=\min\left( |y-1|, |-1-y| \right) $. Again by the Archimedean property by the same argument as before there exists $n\in \mathbb{N}$ with $d\geq \frac{1}{n}$, so $x\notin [-1-\frac{1}{n}, 1+\frac{1}{n} ]$. Thus $B_1\supseteq\bigcap\mathcal A$.

I claim $B_2=(-2, 2)=\bigcup\mathcal A$. Let $x\in (-2, 2)$, then $x\in(-1-1, 1+1)\in\mathcal A$, so $B_2\subseteq\bigcup\mathcal A$. Let $y\in \mathbb{R}\setminus (-2, 2)$. Then for any $n\in \mathbb{N}$, $|y|\geq 1+\frac{1}{n}$, so $y\notin(-1-\frac{1}{n}, 1+\frac{1}{n})$ and $B_2\supseteq\bigcup\mathcal A$.


\end{document}
