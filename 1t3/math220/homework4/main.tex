\documentclass[letterpaper, reqno,11pt]{article}
\usepackage[margin=1.0in]{geometry}
\usepackage{color,latexsym,amsmath,amssymb,graphicx, float}
\usepackage{hyperref}

\hypersetup{
colorlinks=true,
linkcolor=magenta,
filecolor=magenta,
urlcolor=cyan,
}

\graphicspath{ {images/} }

\newcommand{\RR}{\mathbb{R}}
\newcommand{\CC}{\mathbb{C}}
\newcommand{\ZZ}{\mathbb{Z}}
\newcommand{\QQ}{\mathbb{Q}}
\newcommand{\NN}{\mathbb{N}}
\newcommand{\st}{\text{ s.t.}\ }
\newcommand{\tn}[1]{\textnormal{#1}}
\newcommand{\m}{\textnormal{ m}}
\newcommand{\s}{\textnormal{ s}}
\newcommand{\K}{\textnormal{ K}}
\newcommand{\h}{\textnormal{ h}}
\newcommand{\W}{\textnormal{ W}}
\newcommand{\J}{\textnormal{ J}}
\newcommand{\Pa}{\textnormal{ Pa}}
\newcommand{\mol}{\textnormal{ mol}}
\newcommand{\Hz}{\textnormal{ Hz}}
\newcommand{\kg}{\textnormal{ kg}}
\newcommand{\cm}{\textnormal{ cm}}
\newcommand{\mm}{\textnormal{ mm}}
\newcommand{\N}{\textnormal{ N}}

\begin{document}
\pagenumbering{arabic}
\title{Math 220 Homework 4}
\date{October 12, 2021}
\author{Xander Naumenko}
\maketitle

{\noindent\bf Question 1.} We will prove this by induction on $n$. 

{\bf Base case (n=0):} When $n=0$, $n^3+(n+1)^3+(n+2)^3=1^3+2^3=9$ which is divisible by $9$. 

{\bf Inductive step:} Assume the result holds for $n$, i.e. $9|n^3+(n+1)^3+(n+2)^3$. Then plugging in $n+1$, we get 

$$
    (n+1)^3+(n+2)^3+(n+3)^3=n^3+(n+2)^3+9n^2+27n+27\equiv 9n^2+27n+27\mod9=0\mod9
$$

Thus by induction the result holds for every $n$. $\square$ 

{\noindent\bf Question 2.} Let $a, b, c\in\ZZ$ with $\gcd(a, b)=1$ and assume $a|bc$. Then by B\'ezout's identity we have that $\exists x, y\in\ZZ\st ax+by=1$. Also, based on the assumption of divisility we have that $\exists m\in\ZZ\st bc=ma\Rightarrow b=\frac{ma}{c}$. Substituting we get 

$$
    ax+\frac{may}{c}=1\Rightarrow axc+may=c=a(xc+my)
$$

This means that $c$ is divisible by $a$ since $xc+my$ is an integer. $\square$

{\noindent\bf Question 3a.} This statement is false, which can be shown with a counterexample. Let $x=3\in P, y=3\in P$, so $x+y=6\notin P$ which contradicts the statement. $\square$

{\noindent\bf Question 3b.} This statement is false. To show this let $x=7\in P$, and let $y\in P$ be arbitrary. There are two cases: either $P$ is odd or even. If it is even then the only even prime is two, so $x+y=7+2=9\notin P$. If $y$ is odd, then it can be expressed as $y=2m+1$ and we have $x+y=7+2m+1=2(4m+1)$ which can't be prime, since it's divisible by two. In either case the result can't be prime, so the original statement is false. 

{\noindent\bf Question 3c.} This statement is false. Let $x\in P$. If $x=2$ then let $y=2$, so $x+y=2+2=4\notin P$. If $x\neq 2$ then $x$ is odd and let $y=3$. Then we can express $x=2m+1, m\in\ZZ$ and $x+y=2m+1+3=2(m+2)$ which isn't prime since it's divisible by two. Thus the original statement was false. 

{\noindent\bf Question 3d.} This statement is true. To show this choose $x=2$ and $y=3$, so $x, y\in P$ and $x+y=3+2=5\in P$ and we're done. $\square$

{\noindent\bf Question 4.} Let $\epsilon>0$, and let $M=\frac2{\sqrt\epsilon}$. Then for $x\geq M=\frac2{\sqrt\epsilon}$, we have 

$$
    \bigg|\frac{2x^2}{x^1+1}-2\bigg|=\bigg|\frac{-2}{x^2+1}\bigg|< 2\bigg|\frac1{x^2}\bigg|\leq2\bigg|\frac{\epsilon}{4}\bigg|=\frac\epsilon2<\epsilon
$$

This matches the result so we're done. $\square$

{\noindent\bf Question 5.} $f$ is continuous at $x=0$. Let $\epsilon>0$ and choose $\delta=\sqrt\epsilon$. Then $\forall|x|<\delta$, we have 

$$
    |x^2\sin(\frac1x)-0|\leq |x^2|<|(\sqrt x)^2|=|\epsilon|=\epsilon
$$

This means that $\lim_{x\to a}f(x)=0=f(0)$ so $f$ is continuous. $\square$


{\noindent\bf Question 6.} By definition if $(x_n)$ converges to 0, then $\forall \epsilon>0, \exists N\in\NN\st |x_n|<\epsilon\forall n>N$. Let $\epsilon=1$, and using the previously stated definition $\exists N\in\NN\st |x_n|<1\forall n>N$. Since $N$ is finite we can define $M^\prime=\max(x_0, x_1, \ldots, x_N)$. Let $M=\max(M^\prime, 1)$. $(x_n)$ is bounded by $M$ since $\forall n\in\NN$, either $n<N$ and $|x_n|\leq M^\prime\leq M$ or $n>N$ and $|x_n|<\epsilon=1\leq M$. Thus $(x_n)$ is bounded. $\square$

{\noindent\bf Question 7a.} Let $M\in\RR$. If $M\leq1$ then let $t=\frac{1}{e^2}\in(0, 1)$, so $|f(t)|=|\log e^{-2}|=2>1\geq M$. Otherwise let $t=e^{-M-1}\in(0, 1)$ and we have $|f(t)|=|\log e^{-M-1}|=|-M-1|=M+1>M$. In either case it is not bounded so we're done. $\square$

{\noindent\bf Question 7b.} Let $c=\frac34\in(1/2, 3/1)$ and let $M\in\RR$. If $M\leq0$ then the result follows trivially so assume $M>0$. Next let $t=\frac34-1/(4\sqrt{M+1})$. Then we have 

$$
    |f(t)|=\bigg|\frac{(1-t)^2}{(\frac34-1/(4\sqrt{M+1})-\frac34)^2}\bigg|\geq \bigg|\frac{(1/4)^2}{1/(16(M+1))}\bigg|=|M+1|>M
$$

This fulfills the definition so we're done. $\square$

\end{document}