\documentclass[letterpaper, reqno,11pt]{article}
\usepackage[margin=1.0in]{geometry}
\usepackage{color,latexsym,amsmath,amssymb,graphicx, float}
\usepackage{hyperref}

\hypersetup{
colorlinks=true,
linkcolor=magenta,
filecolor=magenta,
urlcolor=cyan,
}

\graphicspath{ {images/} }

\newcommand{\RR}{\mathbb{R}}
\newcommand{\CC}{\mathbb{C}}
\newcommand{\ZZ}{\mathbb{Z}}
\newcommand{\QQ}{\mathbb{Q}}
\newcommand{\NN}{\mathbb{N}}
\newcommand{\st}{\mathrm{ s.t.}\ }
\newcommand{\tn}[1]{\textnormal{#1}}
\newcommand{\m}{\textnormal{ m}}
\newcommand{\s}{\textnormal{ s}}
\newcommand{\K}{\textnormal{ K}}
\newcommand{\h}{\textnormal{ h}}
\newcommand{\W}{\textnormal{ W}}
\newcommand{\J}{\textnormal{ J}}
\newcommand{\Pa}{\textnormal{ Pa}}
\newcommand{\mol}{\textnormal{ mol}}
\newcommand{\Hz}{\textnormal{ Hz}}
\newcommand{\kg}{\textnormal{ kg}}
\newcommand{\cm}{\textnormal{ cm}}
\newcommand{\mm}{\textnormal{ mm}}
\newcommand{\N}{\textnormal{ N}}

\begin{document}
\pagenumbering{arabic}
\title{Math 220 Homework 2}
\date{September 27, 2021}
\author{Xander Naumenko}
\maketitle

{\noindent\bf Question 1.} Prove that if $a\in\ZZ$, then $4\not| a^2+1$. 

\medskip

There are two cases: either $a$ is even or it is odd. If it is even, then it can be represented as $a=2m$, $m\in\ZZ$ and we get 

$$
    a^2+1=4m^2+1
$$

Since $m^2\in\ZZ$, $a^2+1$ is not divisible by four in this case. In the other case, $a$ being odd, it can be represented as $a=2n+1$ and we get 

$$
    a^2+1=(2n+1)^2+1=4n^2+4n+1+1=4(n^2+n)+2
$$

Since $n^2+n\in\ZZ$ this also isn't divisible by four. Thus in all cases $4\not|a^2+1$ and so we are done. $\square$

{\noindent\bf Question 2.} Let $x$ be a positive real number. Prove that if $x-\frac3x>2$, then $x>3$. 

\medskip

Rearranging, we get 

$$
    x-\frac3x>2\Rightarrow x^2-2x-3>0\Rightarrow (x-3)(x+1)>0
$$

Since $x+1$ is always positive (since $x>0$), the only way this is true is if $x-3>0$, i.e. $x>3$ and we're done. $\square$

{\noindent\bf Question 3.} Prove that if $k\in\ZZ$, then $3|k(2k+1)(4k+1)$. 

\medskip

By Euclidean division there are three possible cases: $k=3m, k=3m+1$ or $k=3m+2, m\in\ZZ$. 

{\bf Case $k=3m$:} In this case, we have that the expression in the question becomes

$$
    k(2k+1)(4k+1)=3m(2k+1)(4k+1)
$$

This is divisible by 3 so we're done with this case. 

{\bf Case $k=3m+1$:} Again writing the expression we get 

$$
    k(2k+1)(4k+1)=k(4k+1)(6m+3)=3k(4k+1)(2m+1)
$$

This is divisible by 3 so we're done with this case. 

{\bf Case $k=3m+2$:} Rewriting the expression one last time gives 

$$
    k(2k+1)(4k+1)=k(2k+1)(12m+9)=3k(2k+1)(4m+3)
$$

This is divisible by 3 so we're done with this case. 

Since all cases were consistent with the result the given expression must be divisible by $3$. $\square$

{\noindent\bf Question 4.} Let $n\in\ZZ$. {\bf a.} Show that if $3|n$ and $4|n$, then $12|n$. {\bf b.} Use the previous part to show that if $n>3$ is a prime, then $n^2\equiv 1(\textnormal{mod }12)$. 

\medskip

{\noindent\bf Question 4a.} Using the hypotheses we can write $n=3a, a\in\ZZ$. Since $n$ is also divisible by 4 and $4\not|3$, $a$ must be divisible by 4, i.e. $a=4b, a\in\ZZ$. Thus $n=12b$ and we're done. $\square$

{\noindent\bf Question 4b.} Since $n$ is prime it is not divisible by either 2 or 3. Thus we have $n^2=(2a+1)^2=4a^2+4a+1=4(a^2+a)+1, a\in\ZZ$. Define $m=n^2-1=4(m^2+m)$ Since it also isn't divisible by $3$ we either get $n=3b+1$ or $n=3c+2,b, c\in\ZZ$. In the first case, we get

$$
    n^2=9b^2+6b+1=3(3b^2+2b)+1=m+1
$$

In the second we get that 

$$
    n^2=9c^2+12c+4=3(3c^2+4c+1)+1=m+1
$$

Either way $m$ is divisible by both $3$ and $4$, so by the first part of the question $12|m$. Since $n^2=m+1$ we have that $n^2\equiv1(\textnormal{mod } 12)$. $\square$

{\noindent\bf Question 5.} Prove that if $n^3+n^2-n+3$ is a multiple of three, then $n$ is a multiple of three. 

\medskip

We will use proof by contradiction, so assume that $n^3+n^2-n+3$ is divisible by 3 but $n$ isn't divisible by $3$. We will consider the cases that $n=3m+1$ and $n=3m+2$. 

{\bf Case $n=3m+1$:} Expanding we get 

$$
    n^3+n^2-n+3=27m^3+27m^2+9m+1+9m^2+6m+1-3m-1+3
$$

$$
    =3(9m^3+18m^2+5m)+1
$$

Thus the expression isn't divisible by $3$ which contradicts our assumption it was. 

{\bf Case $n=3m+2$:} Expanding we get 

$$
    n^3+n^2-n+3=27m^3+54m^2+36m+8+9m^2+12m+4-3m+2+3=3(18m^3+21m^2+17m)+2
$$

It is not divisible by $3$, which contradicts our assumption it was. In both cases a contradiction arises, so our original assumption must have been false and $n$ is divisible by 3. $\square$

{\noindent\bf Question 6.} Let $x\in\RR$. Then, prove that $x^2+|x-6|>5$. 

\medskip

Consider three possibilities: $x<1$, $1\leq x\leq 2$, $2< x\leq3$ and $x>3$. 

{\bf Case $x<1$:} In this case we have 

$$
    x^2+|x-6|>0+|1-6|=5
$$

{\bf Case $1\leq x\leq 2$:} In this case we have

$$
    x^2+|x-6|>1+|2-6|=5>5
$$

{\bf Case $2<x\leq 3$:} In this case we have

$$
    x^2+|x-6|>4+|3-6|=7>5
$$

{\bf Case $x>3$:} In this case we have 

$$
    x^2+|x-6|>9+0=0>5
$$

Since this covers all cases we're done. $\square$

{\noindent\bf Question 7.} Let $x, y\in\ZZ$. Prove that $3\not|(x^3+y^3)$ if and only if $3\not|(x+y)$. 

We will prove the contrapositive, i.e. that $3|(x+y)$ if and only if $3|(x^3+y^3)$. For the forward direction of this statement, assume that $3|(x+y)$. Then we have that $x+y=3m, m\in\ZZ$ and 

$$
    x^3+y^3=(x+y)(x^2-xy+y^2)=3m(x^2-xy+y^2)
$$

This is divisible by $3$ so this direction is done. For the reverse direction, assume $x^3+y^3=3m, m\in\ZZ$. Expanding we get that the following is divisible by 3 also: 

$$
    x^3+y^3=(x+y)(x^2-xy+y^2)=(x+y)((x+y)^2-3xy)
$$

This means that either $x+y$ is divisible by three or $(x+y)^2-3xy$ is divisible case (or both, in which the argument for either case works). In the former case the result is shown automatically, and in the latter this implies that $\exists n\in\ZZ$ s.t. $(x+y)^2-3xy=3n\Rightarrow (x+y)^2=3(n+xy)$. Since $(x+y)^2$ is divisible by 3 and 3 is prime this means that $x+y$ must also be divisible by 3, and we're done in both cases. $\square$


\end{document}
