\documentclass[letterpaper, reqno,11pt]{article}
\usepackage[margin=1.0in]{geometry}
\usepackage{color,latexsym,amsmath,amssymb,graphicx, float}
\usepackage{hyperref}

\hypersetup{
colorlinks=true,
linkcolor=magenta,
filecolor=magenta,
urlcolor=cyan,
}

\graphicspath{ {images/} }

\newcommand{\RR}{\mathbb{R}}
\newcommand{\CC}{\mathbb{C}}
\newcommand{\ZZ}{\mathbb{Z}}
\newcommand{\QQ}{\mathbb{Q}}
\newcommand{\NN}{\mathbb{N}}
\newcommand{\st}{\text{ s.t.}\ }
\newcommand{\tn}[1]{\textnormal{#1}}
\newcommand{\m}{\textnormal{ m}}
\newcommand{\s}{\textnormal{ s}}
\newcommand{\K}{\textnormal{ K}}
\newcommand{\h}{\textnormal{ h}}
\newcommand{\W}{\textnormal{ W}}
\newcommand{\J}{\textnormal{ J}}
\newcommand{\Pa}{\textnormal{ Pa}}
\newcommand{\mol}{\textnormal{ mol}}
\newcommand{\Hz}{\textnormal{ Hz}}
\newcommand{\kg}{\textnormal{ kg}}
\newcommand{\cm}{\textnormal{ cm}}
\newcommand{\mm}{\textnormal{ mm}}
\newcommand{\N}{\textnormal{ N}}

\begin{document}
\pagenumbering{arabic}
\title{Math 220 Homework 5}
\date{October 18, 2021}
\author{Xander Naumenko}
\maketitle

{\noindent\bf Question 1.} We will use induction on $n$. 

{\bf Base Case (n=1):} Computing the sum we get 

$$
    \sum_{k=1}^1(2k-1)2^k=(2-1)2=2=6+2(4-6)=6+2^n(4n-6)
$$

{\bf Induction Step:} Assume that $\sum_{n=1}^n(2k-1)2^k=6+2^n(4n-6)$. Then we have 

$$
    \sum_{k=1}^{n+1}(2k-1)2^{k}=6+2^n(4n-6)+(2n+1)2^{n+1}=6+2^n(4n-6)+(4n+2)2^n
$$
$$
    =6+2^n(4n-6)+(4(n+1)-6)2^n=6+(4(n+1)-6)2^{n+1}
$$

This matches the result so we're done. $\square$

{\noindent\bf Question 2.} We will use strong induction on $n$ for $n\geq 3$. 

{\bf Base Case (n=3):} Using our definition for $a_n$ we have 

$$
    a_3=5a_{2}-6a_1=25-6=19=27-8=3^3-2^3
$$

{\bf Induction Step:} Suppose that the result holds for all natural numbers less than $n+1$. Then we have 

$$
    a_{n+1}=5a_n-6a_{n-1}=5(3^n-2^n)-6(3^{n-1}-2^{n-1})=(3^{n+1})-(-2^{n+1})=3^{n+1}+2^{n+1}
$$

This is what we are trying to prove so we're done. $\square$

{\noindent\bf Question 3.} We will use strong induction on $n$. 

{\bf Base Case (n=3):} Using our definition we have 

$$
    a_3=a_2+a_1+a_0=1+3+9=14\leq 27=3^n
$$

{\bf Inductive Step:} Assume that $a_m\leq 3^m$ for all $m\leq n$. Then We have 

$$
    a_{n+1}=a_n+a_{n-1}+a_{n-2}\leq 3^n+3^{n-1}+3^{n-2}\leq 3^n+3^n+3^n=3^{n+1}
$$

Thus the result holds for all $n\geq 3$. $\square$

{\noindent\bf Question 4.} We will use strong induction on $n$. 

{\bf Base Case (n=1):} By explicit definition we have that 

$$
    f_{n+1}f_{n-1}-f_n^2=(1+0)0-1=-1=(-1)^1=(-1)^n
$$

{\bf Inductive Step:} Assume that the result holds for all values less than or equal to $n$, which gives $f_{n+1}f_{n-1}-f_{n}^2=(-1)^{n}$. Then we get 

$$
    f_{n+2}f_{n}-f_{n+1}^2=(f_{n+1}+f_n)f_n-f_{n+1}(f_{n}+f_{n-1})=f_{n+1}f_n+f_n^2-f_{n+1}f_n-f_{n+1}f_{n-1}
$$ 
    
$$
    =-(f_{n+1}f_{n-1}-f_n^2)=-(-1)^n=(-1)^{n+1}
$$

Thus by induction the result holds for all $n\geq 1$. $\square$

{\noindent\bf Question 5.} We will prove the result by induction on $n$. 

{\bf Base Case ($n=0$):} Plugging in we get $7^{0+3}+2=343+2=345=0\mod 5$. 

{\bf Induction Step:} Assume that the result holds for $n$. Then we have that $\exists m\in\ZZ\st 7^{4n+3}+2=5m$ and so

$$
    7^{4(n+1)+3}+2=7^3 7^{4n+3}+2=7^4(5m-2)+2=7^4\cdot 5m-4802+2=5(7^4m-960)=0\mod 5
$$

Thus by induction the result holds for all nonnegative integers $n$. 

{\noindent\bf Question 6.} We will use induction to prove that $a_n$ is increasing and greater than 2 for all $n$. 

{\bf Base Case (n=1):} $a_1=3< 6=9-3=a_1^2-a_1=a_2$, and $a_1=3>2$. 

{\bf Induction Step:} Assume that $a_n>a_{n-1}$ and $a_n>2$. Then because $a_n>2$ then $a_n-1>1$, so 

$$
    a_{n+1}=a_n^2-a_n=a_n(a_n-1)> a_n
$$

Thus the result holds for all $n$. $\square$

{\noindent\bf Question 7.} We will use strong induction on $n$. 

{\bf Base Cases ($n=0$ and $n=1$):} For $n=0$ then we have $u_0=1=\cos(0)=\cos(nx)$. For $n=1$ we get $u_1=\cos x=\cos(nx)$ so the base case is fulfilled. 

{\bf Inductive Step:} Assume that the result holds for all numbers less than $n+1>1$. Then 

$$
    u_{n+1}=2u_1u_{n}-u_{n-1}=2\cos(x)\cos(nx)-\cos((n-1)x)
$$

$$
    =2\cos(x)\cos(nx)-\cos(nx)\cos(x)-\sin(nx)\sin(x)=
$$

$$
    \cos(x)\cos(nx)-\sin(nx)\sin(x)=\cos((n+1)x)
$$

Thus the result holds for all $n$. $\square$

\end{document}