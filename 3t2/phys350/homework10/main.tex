\documentclass[letterpaper, reqno,11pt]{article}
\usepackage[margin=1.0in]{geometry}
\usepackage{color,latexsym,amsmath,amssymb,graphicx, float}
\usepackage{hyperref}

\hypersetup{
colorlinks=true,
linkcolor=magenta,
filecolor=magenta,
urlcolor=cyan,
}

\graphicspath{ {images/} }

\newcommand{\RR}{\mathbb{R}}
\newcommand{\CC}{\mathbb{C}}
\newcommand{\ZZ}{\mathbb{Z}}
\newcommand{\QQ}{\mathbb{Q}}
\newcommand{\NN}{\mathbb{N}}
\newcommand{\st}{\text{ s.t.}\ }

\begin{document}
\pagenumbering{arabic}
\title{MATH 220 Homework 10}
\date{November 28, 2021}
\author{Xander Naumenko}
\maketitle

{\noindent\bf Question 1.} We will use proof by contradiction so suppose not, i.e. suppose that $\exists a\in\NN\st\equiv 2\mod6$ and $a\equiv7\mod9$. Then $\exists m, n\st$

\[
    a=6m+2=9n+7\implies 6m-9n=5\implies 3(2m-3n)=5
\]

Clearly $5$ isn't divisible by $3$ so this is impossible, so the only possibility is that our original assumption was wrong and no such $a$ exists. $\square$

{\noindent\bf Question 2.} As the hint suggests, consider the equation modulo $4$. When doing so we get that 

\[
    y^2\equiv 3\mod 4
\]

There are two possible cases: either $y$ could be even or odd. If it is even then $\exists a\in\ZZ\st y^2=(2a)^2=4a^2\equiv 0\mod 4$, and if it is odd then $\exists b\in\ZZ\st y^2=(2b+1)^2=4(b^2+b)+1\equiv 1\mod4$. In either case it is not possible that $y^2\equiv 3\mod4$ which is a necessary condition for the original equation to have solutions, so no such $x, y\in\ZZ$ exist. $\square$

{\noindent\bf Question 3a.} We will use proof by contradiction so suppose that the inverse wasn't unique. Then there are two functions $f_1^{-1}, f_2^{-1}$ such that they are both inverses of $f$ but $\exists y_1\in Y\st f_1^{-1}(y_1)\neq f_2^{-1}(y_1)$. We proved in class that if $f$ permits an inverse then it must be bijective, and the definition of the inverse implies

\[
    f(f_1^{-1}(y_1))=x_1=f(f_2^{-1}(y_1))
\]

The injectivity of $f$ implies then that $f_1^{-1}(x_1)= f_2^{-1}(x_1)$, but this contradicts the assumption that the two inverses were unique. Thus that assumption must have been incorrect and only one inverse function exists. $\square$

{\noindent\bf Question 3b.} We will first show that $f^{-1}\circ g^{-1}$ is an inverse, and by part a it is also the unique inverse. For any $y\in Y$, we have that 

\[
    f^{-1}\circ g^{-1} (g\circ f(x))=f^{-1}(g^{-1}(g(f(x))))=f^{-1}(f(x))=x
\]

Thus $f^{-1}\circ g^{-1}$ fulfills the definition of being an inverse, so it must be unique by part a. $\square$

{\noindent\bf Question 4.} We will use proof by contradiction, so suppose not. Then $\exists a\in\ZZ, b\in\NN, \gcd(a, b)=1\st \sqrt[3]{25}=\frac ab$. Expanding we get that 

\[
    25=(\frac ab)^3\implies a^3=25b^3
\]

Since $5|25b^3$, this means that $5|a^3$ as well. Since  $5$ is prime this means that $5|a\implies\exists c\in\ZZ\st a=5c$. Plugging this in again we get 

\[
    125c^3=25b^3\implies b^3=5c^3
\]

Using the exact same logic as before $5|b$, but this contradicts our assumption that $gcd(a, b)=1$. Thus our assumption must be wrong and $\sqrt[3]{25}\notin\QQ$. $\square$

{\noindent\bf Question 5.} Proof by contradiction: suppose that it was a perfect square, i.e. suppose that $\exists l\in\NN\st l^2=2n$. Then we would have 

\[
    l=\sqrt{2n}=\sqrt2\sqrt n=m\sqrt2
\]
By assumption $l$ and $m$ are natural numbers, and in class we proved that $\sqrt2\notin\QQ\implies \sqrt2\notin\NN$. Thus the left side is a natural number and the right side isn't, clearly violating equality. The only possibility is that our original assumption was false and $2n$ isn't a perfect square. $\square$

{\noindent\bf Question 6.} To show it is bijective we will show that it is injective and surjective. For surjective, let $m\in\ZZ$. If $m$ is even then choose $n=m-7$ which is odd, and we have that $f(n)=f(m-7)=m-7+7=m$. If $m$ is odd then choose $n=-m-3$ which is even, and we have that $f(n)=f(-m-3)=m-3+3=m$. Thus $\forall m\in\ZZ, \exists n\in\ZZ\st f(n)=m$, so $f$ is surjective. 

For injective, Let $n_1, n_2\in\ZZ$. Suppose $f(n_1)=f(n_2)$, we will show that $n_1=n_2$. There are three cases: the two numbers are both odd, both even or one of each. If they are both even, then we have that 

\[
    f(n_1)=-n_1+3=f(n_2)=-n_2+3\implies n_1=n_2
\]

If they are both odd, then we have that 

\[
    f(n_1)=n_1+7=f(n_2)=n_2+7\implies n_1=n_2
\]

If one is odd and one is even, without loss of generality assume that $n_1$ is the even one. Then we get that 

\[
    f(n_1)=f(2m_1)=-2m+1=f(n_2)=f(m_2+1)=2m_2+8=2(m_2+4)
\]

The left side is odd and the right side is even, so it is not possible that $n_1$ has different parity then $n_2$. Thus all possible cases are either not possible or agree with $n_1=n_2$, so $f$ is injective. Since it is both injective and surjective it is bijective. 

For the inverse, it is the following: 

\[
    f^{-1}(m)=\begin{cases}-m-3&m\text{ odd}\\m-7&m\text{ even}\end{cases}    
\]

To show that this is the case, let $m\in\ZZ$. Then if $m$ is even we have that $m-7$ is odd and

\[
    f^{-1}(f(m))=f^{-1}(m-7)=m
\]

If $m$ is odd then $-m-3$ is even and we have 

    
\[
    f^{-1}(f(m))=f^{-1}(-m-3)=m
\]

Thus $f^{-1}$ is the inverse. $\square$

{\noindent\bf Question 7a.} Expanding we get 

\[
    f\circ f\circ f\circ f(x)=f\circ f(1-\frac1x)=f(1-\frac1{1-\frac1x})=f(1-\frac x{x-1})=1-\frac1{1-\frac x{x-1}}
\]

\[
    =1-\frac{x-1}{-1}=x=i_A
\]

{\noindent\bf Question 7b.} First, note that $i_A=x$ is a bijective function on $A$. This means that for every $y\in A, \exists x\in A\st \exists z\in A\st g\circ g(y)=x$ and $g(x)=y$, which is the definition of being surjective. 

For injectivity, we will use proof by contradiction so suppose $g$ wasn't injective. Then we have that $\exists x_1, x_2\in A\st g(x_1)=g(x_2)$ and $x_1\neq x_2$. Using the identity for $g$ we know that $g(g(g(x_1)))=g(g(g(x_2)))$ and $g(g(g(x_1)))=x_1$ and $g(g(g(x_2)))=x_2$. This would imply that $x_1=x_2$ which contradicts our assumption, so it must be that $g$ is injective as well. Since $g$ is injective and surjective this means that $g$ is bijective. $\square$

{\noindent\bf Question 7c.} By part a $f$ is bijective, so it must have an inverse. the inverse is the following: $f^{-1}(x)=\frac1{1-x}$. Plugging in we have $f^{-1}(f(x))=1-\frac1{\frac1{1-x}}=x$ so it is an inverse, and by question 3 it is the unique inverse. $\square$

{\noindent\bf Question 8.} We will use proof by contradiction, so suppose that such an integer $k$ exists that is rational. Then we would have that $\exists a\in\ZZ, b\in\NN\st \gcd(a, b)=1$ and $\sqrt k=\frac ab$. Then we have that 

\[
    k=\frac{a^2}{b^2}\implies a^2=kb^2
\]

Then $k|a^2$. Since $k$ isn't a perfect square it must have a prime factor $p$ with an odd degree, since if all prime factors had even degrees it would be a perfect square. Then $p|a^2$, and since $p|k$ with odd degree it means that $p|a$ as well. Then we have that $\exists c\in \ZZ\st a=cq$. Then we have that $kb^2=q^2c^2$. Since $k$ is divisible by an odd number of $q$ either $q|b$ or $q|c$. In the latter case we can repeat this process, and since $k$ this process will eventually terminate and the former case will occur. When this happens we have that $q|b$, which contradicts our assumption that $\gcd(a, b)=1$ and thus no such $k$ exists. $\square$




\end{document}